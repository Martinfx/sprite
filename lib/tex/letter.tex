% letter.tex  3 December 1984
\documentstyle{article}


\newcommand\bs{\char '134 }  % A backslash character for \tt font
\newcommand\lb{\char '173 }  % A left brace character for \tt font
\newcommand\rb{\char '175 }  % A right brace character for \tt font


\begin{document}

\begin{center}
\Large Making Letters With \LaTeX
\end{center}

You can now use \LaTeX{} to typeset letters---both personal and
bus\-i\-ness---and their mailing labels.  The standard page layout and
document style to use are both called \hbox{\tt letter}, though there
may be additional page layouts and document styles available at your
location.  Your {\tt .TEX} file starts with the usual commands and
ends with the usual \hbox{\verb"\end{document}"}.

The {\tt letter} document style is designed to make a number of
letters at once.  Consequently, the standard parts of a letter, which
are likely to be the same in all your letters, are defined with
declarations.  The following commands are declarations; they take a
single argument.

\begin{description}
\item[{\tt \bs name}:] Your name, as it should appear in the
   return address on the envelope.  For example,
\begin{verbatim}
       \name{Leslie Lamport}
\end{verbatim}


\item[{\tt \bs address}] The return address, as it should appear
    on the letter and the envelope.  Separate lines of the address
    should be separated by \verb"\\" commands.  For example,
\begin{verbatim}
    \address{2345 Sunnyside Lane \\
             San Francisco, CA 94123}
\end{verbatim}
    If you do not make an
    \hbox{\verb"\address"} declaration, then the letter will
    be formatted for copying onto your organization's standard
    letterhead.  If you give an \hbox{\verb"\address"} declaration,
    then the letter will be formatted as a personal letter.

\item[{\tt \bs signature}:] Your name, as it should appear at the
   end of the letter underneath the space for your signature.
   Items that should go on separate lines should be separated
   by \verb"\\" commands, as in
\begin{verbatim}
    \signature{ Robert Smith \\ Director of 
                Research}
\end{verbatim}


\item[{\tt \bs location}:] This modifies your organization's standard
    address.  For example, it might be a room number.

\item[{\tt \bs telephone}:]  Your telephone number.
\end{description}

These declarations have the usual scoping rules.  Hence, you would
probably make the ``standard'' declarations at the beginning of your
file, with local modifications for the individual letters.  Or, you
could have all your business letters first, then give an
\hbox{\verb"\address"} declaration, followed by your personal letters.
Some of these declarations may be ignored by the document style.

Each letter is a {\tt letter} environment, whose argument
is the name and address of the recipient.  For example, 
you might have
\begin{verbatim}
       \begin{letter}{Mr. John Doe \\ 2345 Jones St. 
           \\ Oakland, CA 91123}
         ...
       \end{letter}
\end{verbatim}

Following the \verb"\begin{letter}" comes any declarations
local to the letter---for example, if you are using a
nonstandard \hbox{\verb"\signature"} for this letter.

The letter itself begins with an \hbox{\verb"\opening"}
command, such as
\begin{verbatim}
       \opening{Dear John,}
\end{verbatim}
The text of the letter follows.  It is typed as ordinary
\LaTeX{} input.  (Commands that make no sense in a letter,
like \hbox{\verb"\chapter"}, don't work.)  The letter
closes with a \hbox{\verb"\closing"} command, like
\begin{verbatim}
       \closing{Best regards,}
\end{verbatim}
which generates the ``Best regards,'' together with the ``signature''.

After the closing, you can have additional material.  This is typed as
usual, except that paragraphs are not indented, regardless of whether
or not they are indented in the main body of the letter.  The 
\hbox{\verb"\cc"} command produces the usual \hbox{``cc: $\ldots$''},
as in:
\begin{verbatim}
      \cc{J. Tinkers \\ R. Evers \\ C. Chance}
\end{verbatim}
There's also a similar \hbox{\verb"\encl"} command for a list of
enclosures.  \LaTeX{} just puts ordinary interparagraph vertical space
between all this end matter, which is usually not enough.  Use the
\hbox{\verb"\smallskip"}, \hbox{\verb"\medskip"}, and
\hbox{\verb"\bigskip"} commands to put in the right amount of space.

The \verb"\ps" command resumes normal formatting in case you want to
add a postscript.  The \verb"\ps" command does not generate any text;
you have to type the ``P.S.'' yourself.  Page breaking, which is
inhibited after the \hbox{\verb"\closing"}, is allowed after the
first line of output following the \verb"\ps" command.

You can get \LaTeX{} to produce mailing labels by typing a 
\hbox{\verb"\makelabels"} command before the \hbox{\verb"\begin{document}"}
command.  \LaTeX{} prints the labels after all the letters, in a
format suitable for xerographic copying onto ``peel-off'' labels.
(The exact format will depend upon the labels available at your
institution.)  A null \hbox{\tt letter} environment---i.e., one
with nothing between the \hbox{\verb|\begin|} and
\hbox{\verb|\end|} commands---will produce nothing but a mailing
label.  Use such null environments to produce the mailing labels
for copies of the letter.

\end{document}
