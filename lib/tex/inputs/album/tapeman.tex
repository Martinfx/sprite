\documentstyle[tape]{article}
\setlength{\textwidth}{6.5in}
\setlength{\textheight}{9in}
\setlength{\oddsidemargin}{1in}
\renewcommand{\baselinestretch}{1.25}
\parindent=15pt
\parskip=6pt plus 1pt
\pagestyle{plain}
\begin{document}

\section*{Making Casette Tape Labels using \LaTeX}
\begin{flushright}
\begin{tabular}{@{}l@{}}
Sunil Podar (podar@sbcs)\\
S.U.N.Y.\ at Stony Brook.\\

\end{tabular}
\end{flushright}

\vspace*{8mm}
This document describes a way to maintain a simple ``album database'' and
susequently print formatted casette tape labels. The commands to accomplish
this have been implemented as a documentstyle option {\tt tape} and should be
used with the {\it article} documentstyle, i.e., it should be invoked with
the command:

\verb|\documentstyle[tape]{article}|

A casette has two sides and each side is assumed to hold one {\it album}.
Each side of a casette tape label has three major fields of information,
depicted in Figure 1.:

\begin{description}
\item[side\hfill] the textual information that goes on the front-flap. This
is where most of the information goes.
\item[title\hfill] the title given to the {\it side} that goes on the side
face of the casette cover.
\item[addendum\hfill] the textual information that goes on the back-flap of
the casette tape cover.
\end{description}

\album{1}{\it side--1}{\it title--1}{\it addendum--1}
\album{2}{\it side--2}{\it title--2}{\it addendum--2}
\begin{figure}[hb]
\begin{center}
\renewcommand{\globaltapeid}{\footnotesize global\\tapeid}
\small
\maketape[tapeid]{1}{2}%
\caption[]{The Layout of a Casette Tape Label and the Terminology Used}
\end{center}
\end{figure}

\newpage
Each side of the tape in the database is defined using the following
command:

{\it
\verb|\album{|album-id\verb|}{|side\verb|}{|title\verb|}{|addendum\verb|}|}

\verb|\album| command defines an album. More precisely, it defines one side
of a casette tape. It takes four arguements: first one
is a unique label assigned to every album, the other three being {\it side},
{\it title} and {\it addendum}.

A tape label holds information for two sides. Following command produces the
tape label:

{\it \verb|\maketape[|tape-id\verb|]{|album1-id\verb|}%
{|album2-id\verb|}[|optional tape title\verb|]|}

The album entry corresponding to {\it album1-id} goes on the left side of the
tape label and those corresponding to {\it album2-id} on the right side. By
default, individual titles are picked up from the album definition and
formatted to appear under their own sides. If it is desired to have a single
title as the case may be if both sides of a tape are by the same composer or
group, then a title may be explicitly specified as the last arguement in
which case individual titles of the two albums, as specified in their
respective \verb|\album| definitions, are ignored.

A tape label also has two minor
fields of information, that go on the side face of the tape label, on either
side of the title. They are:
\begin{description}
\item[globaltapeid\hfill] This goes on the left side of the title, almost
flush with the left edge. This may hold information such as the owner's name.
By default, it is empty and may be changed using a \verb|\renewcommand|.

\item[tapeid\hfill] This is an identification for individual tapes and goes
to the right of the title almost flush with the right edge. 
\end{description}

The width of the box in which the two fields described above are printed, is
controlled by a variable \verb|\tapeidwidth| and may be changed using a 
\verb|\setlength| command. Also, by default, a vertical line is drawn
separating the two boxes from the title region. It may be changed by a
\verb|\renewcommand| on the parameter \verb|\tapeidmarker| --- one may
prefer a $\bullet$ (\verb|$\bullet$|) as the demarcator.

The {\it side} and the {\it addendum} fields are set in a \verb|minipage|
environment, thus any of the regular paragraph formatting commands of \LaTeX\
may be used; various formatting environments, such as {\tt tabbing}, {\tt
tabular}, {\tt itemize}, {\tt enumerate}, etc., can also be used. The other
fields, i.e., {\it title}, {\it globaltapeid} and {\it tapeid} are set in a
\verb|\parbox| thus only \verb|\\| command may be used with no blank lines in
the fields -- a parbox permits only one simple paragraph. Just a reminder: to
get indentation on a line following a \verb|\\| command, use \verb|\hspace*|
instead of \verb|\hspace|; according to the manual, any \verb|\hspace| after
a \verb|\\| is ignored.

On the outside of the left hand edge of the tape label, three vertical lines
are shown. In addition to enforcing a correct height for each of the three
regions, they also serve as an indicator. Their height is invariant and in
the case when more text has been typed in a region than will fit on the
label, they indicate precisely by how much has the height been overshot. They
should, of course, be left out when the label is cut.

No choice of fonts or fontsizes is enforced and one is completely free to
choose these, although it is suggested that a small-sized font (such as those
obtained by \verb|\small|, \verb|\footnotesize| or \verb|\tiny| declarations)
be used since the width of the paragraphs is rather small. Also, if more
information need to be put than will fit, the inter-line spacing may be
reduced using a \verb|\renewcommand| on the parameter \verb|\baselinestretch|.

Two tape labels can be fit on a page, and a \verb|\newpage| declaration ought
to be placed after every two \verb|\maketape| commands.

It is assumed that each \verb|\album| describes one {\it side} of the tape
label, which is also usually the case. If one side of a tape contains more
than one album then one will have to declare it as a single \verb|\album|
description. On the other hand, if a single album, or piece of music spans
both sides of a casette tape, its description will have to be broken
into two \verb|\album| declarations.

The label can be cut out along the outer boundary -- the border lines are
meant to be part of the label so the cut should be just along the outer edge
of the lines.

\section*{General Comments}

The tape database is implemented by storing all fields as control sequences,
thus it is suggested that the database be constructed using reasonable-sized
files; subsequently to make tape labels, only the necessary files be
\verb|\input|, otherwise \TeX\ may run out of memory.

The tape database has been implemented with the aim that the database
files be sharable amongst various users so long as every album has a unique
{\it album-id}. In order to load a tape database, all one needs to do is
\verb|\input| the appropriate files and subsequently type the
\verb|\maketape| commands for each desired pair of albums.
Following suggestions are made for conventions to be adopted
so as to enable a large community to share files without much effort:
\begin{itemize}
\item Each album database file have a ``{\tt .bum}'' file-name
extension\footnote{music album $\Leftrightarrow$ musical bum --- Benny Hill}.
For such files, the extension will have to be explicitly specified in the
\verb|\input| statement, otherwise \TeX\ will look for a file with a {\tt
.tex} extension.

\item Separate files be maintained for different composers or groups.

\item Any character can be used in the {\it album-id}, thus following
formats are suggested for {\it album-id}'s\\[1mm]
\hspace*{0.5in}\verb|filename.#|, or\\
\hspace*{0.5in}\verb|filename[#]|,  where {\tt \#} is an integer.

For example, one may have a file {\tt mozart1.bum} with various album
descriptions of Mozart's compositions, each having an album-id of the form 
{\tt mozart1.1}, {\tt mozart1.2}, and so on. If a single piece of music
spans both sides then we might label the two parts using another level of
subscripts, for example {\tt mozart.12.1} and {\tt mozart.12.2}.
\end{itemize}

The only problem with above conventions is that if two users have a file with
identical album-id's then the file will have to be
copied and the album-id's changed.

Following page contains an example. Commands used to generate the label would
be:
\leftmargini 25pt
\begin{quote}
\begin{verbatim}
\documentstyle[tape]{article}
\renewcommand{\globaltapeid}{\footnotesize\sf Sunil\\Podar}
\begin{document}
\input{example.bum}
\maketape[\sf WC\\12]{vivaldi.1}{vivaldi.2}
\end{document}
\end{verbatim}
\end{quote}

\newpage
\vspace*{-4mm}
{%
% tape document style  for LaTeX 2.09
% podar@sbcs (Sunil Podar) June 20,1986
% Dept. of Applied Math., SUNY at Stony Brook
% You may use this file for whatever purpose you wish. Please leave this
% notice, identifying original author, intact.

\typeout{Document Style 'tape'. Released 20 June 1986}

% This documentstyle is for making casette tape labels. Two labels can be
% fit on a page with a \newpage command after every two labels.
% It should be invoked with \documentstyle[tape]{article}.
% --- should have made a standalone documentstyle, but was lazy; would
%     involve unnecessary dirty work.
% If more stuff needs to be fit on a label, reduce the \baselinestretch in
% the document and/or use smaller fonts.
% Two commands are implemented:
% \album{album-id}{stuff for the side}{title}{addendum that goes on backflap}
% \maketape[tape-id]{album-id}{album-id}[optional explicit title]
%---------------------------------------------------------------------------
\makeatletter
\textwidth 5in
\textheight 9in
\pagestyle{empty}
\renewcommand{\baselinestretch}{0.85}
\parskip 0pt plus 1pt
%\parindent=0pt %% done inside \maketape
\hbadness=3000% will not complain about every little underfull hbox.
%
\leftmargini 12pt
\leftmarginii 10pt % more than two level deep not expected!
\labelsep 3pt
\leftmargin\leftmargini
\labelwidth\leftmargini\advance\labelwidth-\labelsep
\topsep 2pt plus 1pt minus 1pt
\partopsep 2pt plus 1pt minus 1pt
\parsep 1pt
\itemsep 1pt

%% many paragraphing environments use list's and fiddle with the parameter
%% values, thus need to include them in the \def of \@listi.
\def\@listi{\leftmargin\leftmargini
   \labelwidth\leftmargini\advance\labelwidth-\labelsep
   \topsep 2pt plus 1pt minus 1pt
   \partopsep 2pt plus 1pt minus 1pt
   \parsep 1pt
   \itemsep 1pt}

%%removed the \def of \@listi from \small and \footnotesize
%% --- causes problems when used in center environment.
\def\small{\@setsize\small{11pt}\ixpt\@ixpt
\abovedisplayskip 8.5pt plus 3pt minus 4pt%
\belowdisplayskip \abovedisplayskip
\abovedisplayshortskip \z@ plus2pt%
\belowdisplayshortskip 4pt plus2pt minus 2pt}

\def\footnotesize{\@setsize\footnotesize{9.5pt}\viiipt\@viiipt%------
\abovedisplayskip 6pt plus 2pt minus 4pt%
\belowdisplayskip \abovedisplayskip
\abovedisplayshortskip \z@ plus 1pt%
\belowdisplayshortskip 3pt plus 1pt minus 2pt}
%
%% following two redefined to get the label flushleft.
\def\enumerate{\ifnum \@enumdepth >3 \@toodeep\else
      \advance\@enumdepth \@ne
            \edef\@enumctr{enum\romannumeral\the\@enumdepth}\list
	          {\csname label\@enumctr\endcsname}{\usecounter
		          {\@enumctr}\def\makelabel##1{\rlap{##1}\hss}}\fi}

\def\itemize{\ifnum \@itemdepth >3 \@toodeep\else \advance\@itemdepth \@ne
\edef\@itemitem{labelitem\romannumeral\the\@itemdepth}%
\list{\csname\@itemitem\endcsname}{\def\makelabel##1{\rlap{##1}\hss}}\fi}

%----------------
\setlength{\doublerulesep}{1mm}
\setlength{\arrayrulewidth}{0.8pt}
\newdimen\tapeidwidth
\newdimen\@titlewidth
%------------------------------------------------------------------------
% Here are the user controllable parameters, specific to \makelabel:
% \globaltapeid goes on the little box on the left side of title. Typical
% thing to put there would be name. e.g.
% \renewcommand{\globaltapeid}{\footnotesize\sf Sunil\\Podar}
\def\globaltapeid{} %changed via a \renewcommand
\def\tapeidmarker{\vline} %changed via a \renewcommand
\tapeidwidth=0.35in %changed via a \setlength{\tapeidwidth}{..} command
% We put a \rule of 0in width and appropriate height along with the
% \tapeidmarker for the \vline to work since it is in a makebox[0in].
%-----------------------------------------------------------------------
% \album{5}{side}{title}{addendum} => defines \album@v@s = side and
% \album@v@t = title and \album@v@a = addendum
% the title is the one used by \maketape as default.
%
\long\def\album#1#2#3#4{%
\@namedef{album.#1.s}{#2}% side
\@namedef{album.#1.t}{#3}% title
\@namedef{album.#1.a}{#4}}% addendum
%------------------------------------------------------------------
%\maketape[tape-id]{album-id1}{album-id2}[explicit title 
% formatted as a centered one entry]
% -- makes a tape label of albums identified by album-id1 andalbum-id2. The
% titles are picked up from the \album definition; if an explicit centered
% label is desired then specify it as the third (optional) argument.
%
\def\maketape{\@ifnextchar[{\@maketape}{\@maketape[]}}
\def\@maketape[#1]#2#3{\@ifnextchar[%
{\@imaketape[#1]{#2}{#3}}{\@imaketape[#1]{#2}{#3}[@]}}
%
\def\@imaketape[#1]#2#3[#4]{{\parindent=0pt
\tabcolsep=1pt %want explicit control on spacing
\begin{tabular}{|l@{\hspace{0.1in}}|@{\hspace{0.1in}}l|}
\hline
\makebox[0in][l]{\hskip-2mm\rule[-2.32in]{0.4mm}{2.5in}}%
\ \minipage[t]{1.84in}\@nameuse{album.#2.s}\endminipage &%
\minipage[t]{1.84in}\@nameuse{album.#3.s}\endminipage\ \\
\hline\hline
%
\multicolumn{2}{|@{}l@{}|}{%
\makebox[0in][l]{\hskip-1.8mm\rule[-0.2in]{0.4mm}{0.46in}}%
\,\parbox{\tapeidwidth}{\raggedright\globaltapeid\ }%
\makebox[0in]{\tapeidmarker\rule[-0.2in]{0in}{0.46in}}%
\@titlewidth=1.97in \advance\@titlewidth by-\tapeidwidth%
\ifx#4@%
\parbox{\@titlewidth}{\centering \@nameuse{album.#2.t}\ }%
\parbox{\@titlewidth}{\centering \@nameuse{album.#3.t}\ }%
\else\@titlewidth=2\@titlewidth\parbox{\@titlewidth}{\centering #4\ }%
\fi%
\makebox[0in]{\tapeidmarker\rule[-0.2in]{0in}{0.46in}}%
\parbox{\tapeidwidth}{\raggedleft #1\ }\,}\\
%
\hline\hline
\makebox[0in][l]{\hskip-2mm\rule[-0.72in]{0.4mm}{0.9in}}%
\ \minipage[t]{1.84in}\@nameuse{album.#2.a}\ \endminipage &%
% need \ in minipage to enforce size in case any of the minipage args are null
\minipage[t]{1.84in}\@nameuse{album.#3.a}\endminipage\ \\
\hline
\end{tabular}%
\par\vspace*{0.2in}}}
%------------------------------------------------------------------
\makeatother

\input{example.bum}
\renewcommand{\globaltapeid}{\footnotesize\sf Sunil\\Podar}
\begin{center}
\small
\maketape[\sf WC\\12]{vivaldi.1}{vivaldi.2}
\end{center}%
}

\renewcommand{\baselinestretch}{0.85}
\small\normalsize

\noindent The file {\tt example.bum} contains:
\begin{quote}
\begin{verbatim}
\album{vivaldi.1}{\small Vivaldi, Antonio\hfill (1678-1741)\\
\begin{enumerate}
\item Concerto in G Major for Two Mandolins, Strings \& Basso Continuo
\item Concerto in C Major for Mandolin, Strings \& Basso Continuo
\item Concerto in C Major for Two Mandolins, Two Theorboes,  Two flutes,
Two ``Salmoe'', Two Violins, Cello, Strings \& Basso Continuo
\item Concerto a dve Chori in B flat Major, ``Con Violino Discordato'',
Strings \& Basso Continuo
\end{enumerate}
}%
{\normalsize\sf Vivaldi\\Mandolin Concertos}%
{\small {\it I Solisti Veneti} --- Claudio Scimone\\[1mm]
Bonifacio Bianchi, Alessandro Pitrelli --- Mandolins
}
%%-----------------------------------------------------------
\album{vivaldi.2}{\small Vivaldi, Antonio\\[1mm]
{\bf \underline{The Four Seasons\rule[-0.7mm]{0mm}{1mm}}}

\begin{enumerate}
\item SPRING, Concerto No. 1 in E Major
\item SUMMER, Concerto No. 2 in G Minor
\item AUTUMN, Concerto No. 3 in F Major
\item WINTER, Concerto No. 4 in F Minor
\end{enumerate}

Simon Standage --- Violin\\
Trevor Pinnock --- Harpsichord\\[1mm]
{\it The English Chamber Orchestra} on authentic instruments.
}%
{\normalsize\sf Vivaldi\\The Four Seasons}{}
%%-----------------------------------------------------------
\end{verbatim}
\end{quote}

\end{document}
