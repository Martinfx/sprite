%
% Copyright 1990, Daniel R. Greening.  Copying is permitted for any
% non-commercial purpose as long as this copyright and the
% documentation are retained on copies and derivative works.  No
% warrantee is expressed or implied, and the author makes no claims
% about the usefulness, applicability, or correctness of the program.
%
% Greening Phonelist TeX Format.
%
% This thing is pretty cool, if I do say so myself.  I've used it for
% several years.  It formats phone numbers into a 9x9 block that looks
% like the picture that follows.  The picture is simplified.  There is
% enough room for decent sized names, long phone numbers (with area
% or country-codes and extensions), and addresses with 9-digit
% zipcodes.  Print out the phone list example included and you'll
% see.
%
%     ------------------+------------------+------------------
%    | sally W 784-7861	| dan   W 784-7861 | zeke  W 784-7861 |
%    |       H 271-2629 |       H 271-2629 |       H 271-2629 |
%    |   12 Foster Ct   |   12 Foster Ct   |   12 Foster Ct   |
%    |   New York,NY 10 |   New York,NY 10 |   Topeka,KS 1054 |
%    | georgeW 444-adff |                  | clem  W 444-adff |
%    | schoolW 277-2222 |   KANSAS         | jethroW 277-2222 |
%    | abe   W 331-3333 |                  | abigaiH 331-3333 |
%    | ron   W 231-2341 | jeb   W 231-2341 | emmy  W 231-2341 |
%    |  		|		   |	              |
%     ------------------+------------------+------------------
%    |   MICHIGAN       | dan   W 784-7861 |                  |
%    | dan   H 463-5788 |       H 271-2629 |   WASHINGTON     |
%    |   5582 Golfridge |   12 Foster Ct   |                  |
%    |   Alma, MI,48801 |   Lansing,MI 432 | nancy W 234-2223 |
%    |       W 623-3330 | georgeW 444-adff | georgeW 444-adff |
%    | schoolW 277-2222 | schoolW 277-2222 | schoolW 277-2222 |
%    | abe   W 331-3333 | abe   W 331-3333 | abe   W 331-3333 |
%    | ron   W 231-2341 | ron   W 231-2341 | ron   W 231-2341 |
%    |			|		   |	              |
%     ------------------+------------------+------------------
%    | jeff  W 784-7861 |       W 784-7861 | jack  W 784-7861 |
%    |       H 271-2629 |       H 271-2629 |       H 271-2629 |
%    |   12 Foster Ct   |   12 Foster Ct   |   12 Foster Ct   |
%    |   New York,NY 10 |   New York,NY 10 |   New York,NY 10 |
%    | georgeW 444-adff | georgeW 444-adff | georgeW 444-adff |
%    | schoolW 277-2222 | schoolW 277-2222 | schoolW 277-2222 |
%    | abe   W 331-3333 | abe   W 331-3333 | abe   W 331-3333 |
%    | ron   W 231-2341 | ron   W 231-2341 | ron   W 231-2341 |
%    |			|		   |		      |
%     ------------------+------------------+------------------
%
% I typically take the output and cut it along the horizontal dividing
% lines, then staple the thing together on the left.  If you fold it
% in half, it fits in a pocket or a largish wallet.  If you want to
% stick it in a smaller wallet (and you are patient), you can tape the
% horizontal strips end-to-end and fan-fold it.  It then occupies the
% space of a credit-card.
%
% HOW TO USE:
%
% Precede lines that contain a country, state or province header with 
%  ``$''.  This is an example header for the sovereign state of California:
%
%    $ CALIFORNIA
%
% Lines which contain a phone number work like this:
%
%    person-name ; cat ; phone-number
%
% The ``cat'' is typically a one or two letter abbreviation, as in
% ``W'' for work, ``H'' for home, ``FX'' for FAX, ``P'' for parents,
% etc.  The phone-number field defined in this macro package is large
% enough to accommodate the number ``999-999-9999x9999''.  Here is an
% example: 
%
%    Alan Turing      ; W ; 213-825-2266
%                     ; H ; 914-784-7861
%
% Finally, you indicate an address with the prefix ``>''.  This simply
% indents the text which follows by one \quad.  Here's a bigger
% example:
%
%    $ CALIFORNIA
%
%    Alan Turing      ; W ; 213-825-2266
%    > UCLA Computer Science Dept.
%    > Los Angeles, CA 90024-1647
%
%                     ; H ; 914-784-7861
%    > 1243 Blenheim Lane
%    > Los Angeles, CA 90025
%    
% Blank lines essentially mean nothing when the occur between these
% three constructs.
%
% If you want to include formatted paragraphs, you can do so, but you
% must terminate them with ``\par'' or with a blank line.
%
% Enjoy.  If you make any interesting modifications, let me know.
%
% Dan Greening / dgreen@cs.ucla.edu
%
\font\rm=cmr5
\font\bf=cmbx5
% The following produces 3 column, 3 row output.
\hsize=2.0in
\vsize=3.0in
\voffset=-0.75in
\hoffset=-0.5in
\newdimen\fullhsize\global\fullhsize=7.5in
\newdimen\fullvsize\global\fullvsize=10.5in
\newdimen\hboxsize\global\hboxsize=\hsize\advance\hboxsize by 0.5in
\newdimen\vboxsize\global\vboxsize=\vsize\advance\vboxsize by 0.5in
\def\fullhbox{\hbox to\fullhsize}
\def\fullvbox{\vbox to\fullvsize}
\newcount\colcount \global\colcount=11
\newcount\rowcount \global\rowcount=5
\baselineskip=6 pt plus 1pt
\parindent 0pt
\newcount\boxnum
%
% Define argument to accept ^^M as a terminator for a macro (such as
% ;, $, and >.
%
\def\parsearg #1{\let\next=#1\begingroup\obeylines\futurelet\temp\parseargx}
\def\parseargx{%
\ifx \obeyedspace\temp \aftergroup\parseargdiscardspace \else%
\aftergroup \parseargline %
\fi \endgroup}
{\obeyspaces %
\gdef\parseargdiscardspace {\begingroup\obeylines\futurelet\temp\parseargx}}
\gdef\obeyedspace{\ }
\def\parseargline{\begingroup \obeylines \parsearglinex}
{\obeylines %
\gdef\parsearglinex #1^^M{\endgroup \next {#1}}}
%
\def\initcolbox#1 {\global\setbox#1=\vbox to\vboxsize{\hrule\vss\hbox %
to\hboxsize{\vrule height\vboxsize\hfil\vrule height\vboxsize}\vss\hrule}}
\def\initcolboxes{ \initcolbox11  \initcolbox12  \initcolbox13 \initcolbox14
                   \initcolbox15 \initcolbox16 \initcolbox17
		   \initcolbox18 \initcolbox19 }
\def\outputpage{\shipout\fullhbox{\fullvbox{\box11\vss\box14\vss\box17}\hss
                                  \fullvbox{\box12\vss\box15\vss\box18}\hss
                                  \fullvbox{\box13\vss\box16\vss\box19}}}
\initcolboxes
\output={
  \global\setbox\colcount=\vbox to \vboxsize
       {\hrule width\hboxsize\vss\hbox to \hboxsize
           {\vrule height\vboxsize\hss\columnbox\hss\vrule height\vboxsize}
           \vss\hrule width\hboxsize}
% Current column is now saved.
  \message{(\the\colcount)}
  \ifnum\colcount=19
        \global\colcount=11
        \outputpage
        \advancepageno
	\initcolboxes
     \else
        \global\advance\colcount by 1
     \fi
% For the very end.
  \ifnum\outputpenalty>-100000 \else
     \supereject\outputpage\fi
}
\def\columnbox{\vbox to\vboxsize{\vfill\hbox{\vbox to\vsize{\unvbox255\vfill}}\vfill}}
%
% Here we define ``$'' ``>'' and ``;'' as active macro
% characters.  They serve as separators for fields in our phone list.
%
\rm
  \newdimen\widthb
  \global\setbox0=\hbox{WE}
  \widthb=\wd0
  \newdimen\widthc
  \global\setbox0=\hbox{999-999-9999x9999}
  \widthc=\wd0
%
\catcode`\;=\active
{
  \catcode`\;=11
  \gdef\phonelineyyy #1;#2;#3\finish{
       \catcode`\;=\active\hfill
       \hbox to\widthb{\hfil\ignorespaces #1\hfil}
       \hbox to 0.05in{\hfil}
       \hbox to\widthc{\ignorespaces #2\hfil}
       \par}
  \gdef\phoneline #1{\catcode`\;=11\phonelineyyy #1;;;\finish}
}
\outer\def;{\catcode`\;=11\parsearg\phoneline}
%
\catcode`\>=\active
\def\addressline#1{\hbox to\hsize{\quad\ignorespaces #1\hfil}\par}
\def>{\parsearg\addressline}
%
\catcode`\$=\active
\def\arealine#1{\vbox to 1pt{\vfil}\hbox to\hsize{\hfil{\bf\ignorespaces #1}\hfil}\vbox to 1pt{\vfil}}
\def${\parsearg\arealine}
