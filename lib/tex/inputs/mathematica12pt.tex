% File mathematica12pt.tex -- LaTeX version is mathematica.sty
% Macros for TeX/Mathematica documents
% Written 2/26/1991 by Dan Dill dan@chem.bu.edu


% Copyright (C) 1991 Dan Dill
% This is part of TeX/Mathematica
%
% TeX/Mathematica is distributed in the hope that it will be useful, but
% WITHOUT ANY WARRANTY.  No author or distributor accepts responsibility to
% anyone for the consequences of using it or for whether it serves any
% particular purpose or works at all, unless he says so in writing.
%
% Everyone is granted permission to copy, modify and redistribute this
% file, provided:
%  1.  All copies contain this copyright notice.
%  2.  All modified copies shall carry a prominant notice stating who
%      made the last modification and the date of such modification.
%  3.  No charge is made for this software or works derived from it.  
%      This clause shall not be construed as constraining other software
%      distributed on the same medium as this software, nor is a
%      distribution fee considered a charge.
%
% The right to distribute this file for profit or as part of any commercial
% product is specifically reserved for the author.



%	\begin{mathematica}
%	...
%	\end{mathematica}
%
%  will indent enclosed text by dimension \@mathindent (e.g. 2\parindent) from
%  current leftmargin.  Set with
%
%	\mmaindent{...}	% Default is \parindent
%
%  Verbatim text set in cmtt9.  The macro turns blanks into
%  spaces, starts a new line for each carrige return (or sequence of 
%  consecutive carriage returns), and interprets EVERY character literally.
%  I.e., all special characters \, {, $, etc. are \catcode'd to 'other'.
%
%  Tab characters are expanded to eight character stops.

\catcode`\@=11\relax

% Fonts and sizes
% Don't define \mmasize, since line-spacing might also be adjusted if font changes
\font\cellf@nt=cmtt9 % Font for cell text
\def\v@rbsize{\baselineskip 9pt \parskip 0pt plus 1pt \cellf@nt}
\font\b@xfont=cmtt8 % Font for cell start indicator box
\newbox\t@bbox \setbox\t@bbox=\hbox{\b@xfont\ }
\newdimen\w \w=8\wd\t@bbox% tab amount

% \begin{mathematica}
% This scheme is awfully restrictive, to just mma!
\def\@mmatok{mathematica}
\def\begin#1{\def\testtok{#1}
	\ifx\testtok\@mmatok
		\let\next = \@mma
	\else
		\errmessage{Environment #1 undefined.}
		\let\next = \relax
	\fi
	\next}

\input mathematica

\catcode`\@=12\relax
