% Copyright (C) podar@sbcs (Sunil Podar) July 14,1986.
\documentstyle[11pt,fleqn,epic,misc]{article}
% Archiver's note: I commented the following out to get a standard
% page layout. You will get overfull hboxes though. [KY]
%\setlength{\topmargin}{1cm}
%\setlength{\headheight}{1cm}
%\setlength{\footskip}{2cm}
%\setlength{\oddsidemargin}{1in}
%\setlength{\textwidth}{6.5in}
%\setlength{\textheight}{8.5in}
%\setlength{\parindent}{0pt}
\setlength{\unitlength}{1mm}
\parskip=6pt plus 1pt minus 1pt
\itemsep=0pt
\renewcommand{\baselinestretch}{1.3}
\newcommand{\bs}{\char '134 }  % A backslash char for \tt font
\newcommand{\lb}{\char '173 }  % A left brace char for \tt font
\newcommand{\rb}{\char '175 }  % A right brace char for \tt font
\newcommand{\plotchar}{\makebox(0,0){\large $\otimes$}}
\pagestyle{plain}
\begin{document}
\thispagestyle{empty}
\begin{center}
{\Large\bf Enhancements to the Picture Environment of \LaTeX}
\end{center}

\vspace*{0.5in}
\begin{center}
\begin{it}
Sunil Podar\\
Dept. of Computer Science\\
S.U.N.Y. at Stony Brook\\
Technical Report 86-17\\[3mm]
Version 1.2: July 14, 1986.
\end{it}
\end{center}

\vspace*{1.3in}
\begin{center}
{\bf Abstract}
\end{center}

{\renewcommand{\baselinestretch}{1.5} \small\normalsize
\begin{quote}
This document describes some new commands for the picture environment of
\LaTeX. Some of the picture drawing commands of
\LaTeX\ are very low-level. New higher-level commands are implemented and
described here. These commands enhance the graphic capabilities of
\LaTeX\ and provide a friendlier and more powerful user
interface than currently existent. Their implementation has been done
with the aim of reducing the amount of
manual calculations required to specify the layout of {\it objects}.
With the addition of the commands described in this document, it should be
possible to draw
more sophisticated pictures with lesser effort than was previously possible.
\end{quote}
}

\newpage
\setcounter{page}{1}
\section*{Enhancements to the Picture Environment of \LaTeX}

\bigskip
\section{Introduction}


\LaTeX\ provides a reasonably powerful picture drawing capability.
There are many useful commands provided although the
user-interface has room for improvement. The commands described
in this document aim to achieve a simpler and more powerful interface.

Most picture drawing commands require explicit specification of coordinates
for every {\it object\/}. Although explicit coordinates is the basis of the
picture environment, it is possible to provide higher level commands which
reduce the amount of coordinates that need to be manually calculated. There
are basically two approaches that can be taken in designing such commands:
%
\begin{itemize}
\item providing ability to specify a set of objects such that
the entire set can be plotted by specifying one or two coordinate pairs;
\verb|\shortstack| command falls into this category.

\item providing commands that do most of the computation internally and
require simple coordinate pairs to be specified; \verb|\multiput| command is
one example of this approach.

\end{itemize}

The obvious advantage of having commands that fall into the above categories
is that not only they are easier to specify initially, but any subsequent
modification to the layout requires minimal recalculations. For instance, to
modify the coordinates in a \verb|\multiput| statement plotting $n$ objects
requires recalculation of at most 4 coordinates, whereas the equivalent
\verb|\put| statements may require upto $2n$ calculations and/or
recalculations.

Another frequently used command, \verb|\line| has severe limitations and
drawbacks. The
arguments that the \verb|\line| command expects are very non-intuitive and
requires extensive calculations --- often the thought process in writing a
\verb|\line| command involves:
\begin{enumerate}
\item calculating the coordinates of the two end-points.
\item calculating the horizontal and vertical distance.
\item figuring out if the desired slope is available and if not then
repeating steps 1 and 2 till a satisfactory slope is achieved.
\item translating above into an (x,y) pair for specifying a slope and a
horizontal distance for specifying the length of the line.
\end{enumerate}

Above mechanism is a cumbersome way of specifying a line. It also has the
drawback that the length of the shortest line of different slopes that
can be drawn is different; for instance, assuming \verb|\unitlength=1pt|,
\verb|\line(1,6){10}| is the shortest line of the given slope that can be
drawn; it is considerably longer than the available line segment of this
slope --- 60.8pt rather than about 11pt. It should be emphasized that this
is a drawback of only the implementation of the \verb|\line|
command and is not an inherent limitation. This report describes a few line
drawing commands all of which overcome such a drawback, while providing a
simpler syntax. They all take, as arguments, only the coordinates of the
end-points, thus eliminating all other steps involved in specifying a
line; it also seems to be a natural way of perceiving a line in an
environment where all the work is done in terms of coordinates.

A few new commands are developed and described in this report. They provide
a simpler syntax and a higher-level user-interface. Also some of the commands
permit one to plot objects that were previously cumbersome or difficult to
plot. All existing commands still remain accessible. With the new
commands it should now be possible to make pictures with less effort and
make more sophisticated pictures than was possible earlier.

\section{Commands}

Following commands are described here:
\begin{verbatim}
     \multiputlist     \dottedline     dottedjoin environment     \jput
     \matrixput        \dashline       dashjoin environment       \picsquare
     \grid             \drawline       drawjoin environment       \putfile
\end{verbatim}

All the examples in the following sections have been plotted with
\verb|\unitlength = 1mm|.

\subsection{{\tt \bs}multiputlist}

{\tt SYNOPSIS:\\
\hspace*{\leftmargin}%
\bs multiputlist($x$,$y$)($\Delta x$,$\Delta y$)[tbrl]%
\lb{\it item1},{\it item2},{\it item3},\ldots,{\it itemN}\rb}

This command is a variation of the regular \LaTeX\ command \verb|\multiput|.
The \verb|\multiput| command permits one to put the {\it same\/} object
at regularly spaced coordinates. Often one wishes to put {\it different\/}
objects at coordinates that have regular increments -- \verb|\multiputlist|
command can be used in those cases. This command enables one to specify a
collection of objects with a single command thus simplifying the task of
calculating coordinates. All those objects may also be plotted separately
using \verb|\put| commands, but any future revision of those coordinates
may involve lot of manual work. This command also encourages certain
regularity and symmetry in laying out various objects in a picture.

In the \verb|\multiputlist|, as the coordinates are incremented, the objects
to be put are picked up from the {\it list of items\/},
i.e., first item in first position, second item in second position, and so on.
For example, numbers along the X-axis in a graph may be plotted by simply
specifying:\\[2mm]
\hspace*{\leftmargin}%
\verb|\multiputlist(0,0)(10,0){1.00,1.25,1.50,1.75,2.00}| \\[2mm]
This is almost equivalent to the sequence:
\begin{quote}
\begin{verbatim}
\put(0,0){1.00}
\put(10,0){1.25}
\put(20,0){1.50}
\put(30,0){1.75}
\put(40,0){2.00}
\end{verbatim}
\end{quote}

The difference is that each {\it item\/} is put in a
\verb|\makebox(0,0)[tbrl]{...}| kind of construction which 
allows the specification of
the reference point of the box containing the item. The \verb|[tbrl]|
is optional and its absence makes the item centered at the specified
coordinate. Note that \verb|\put| command does not have such an option.

The objects in the {\it list\/} can be virtually anything including any
\verb|\makebox|, \verb|\framebox|, math characters, etc. This command can be
usefully employed in a situation where a variety of objects are
to be put at coordinates that have a regular increment along the x-axis and
the y-axis.

Few comments about \verb|\multiputlist| command:

\begin{itemize}
\item Individual items have to be grouped in \verb|{}| if they contain ``,''s.

\item In the list of items, blanks are not ignored (of course, consecutive
blanks are coalesced into one, as always). For a list of items longer than a
line of input, put a \% at the end in order to nullify the newline if a
blank is not intended to be a part of the item.

\item Specifying individual items in a list format provides a powerful
mechanism for specifying a variety of objects in a single command.
Moreover, often real numbers need to be plotted and
it is nontrivial to generate real numbers or otherwise
handle them in \TeX; they need to be explicitly specified as {\it objects} in
the desired format.
The \verb|\multiputlist| command somewhat simplifies such a task.

\item The implementation of \verb|\multiputlist| uses two macros derived from
the ones given in the \TeX book, namely, \verb|\lop| and \verb|\lopoff| for
list-manipulation.
\end{itemize}

\subsection{{\tt \bs}matrixput}
{\tt SYNOPSIS:\\
\hspace*{\leftmargin}%
\bs matrixput($x$,$y$)($\Delta x_1$,$\Delta y_1$)\lb $n_1$\rb
($\Delta x_2$,$\Delta y_2$)\lb $n_2$\rb \lb {\it object}\rb}

Above command is the two-dimensional equivalent of the regular \LaTeX\ command
\verb|\multiput|. The \verb|\matrixput| command is equivalent to:

\begin{quote}
{\tt \bs multiput($x$,$y$)%
($\Delta x_2$,$\Delta y_2$)\lb $n_2$\rb\lb {\it object}\rb}\\
{\tt \bs multiput($x + \Delta x_1$,$y + \Delta y_1$)%
($\Delta x_2$,$\Delta y_2$)\lb $n_2$\rb\lb {\it object}\rb}\\
\ldots\\
{\tt \bs multiput($x + n_1 \Delta x_1$,$y + n_1\Delta y_1$)%
($\Delta x_2$,$\Delta y_2$)\lb $n_2$\rb\lb {\it object}\rb}
\end{quote}

However, it is more efficient to use \verb|\matrixput| than the equivalent
$n_1$ \verb|\multiput| statements; first the objects along the dimension with
larger index are saved in a box and subsequently the box is copied along the
other dimension, resulting in a $O(n_1 + n_2)$ execution time rather than
$O(n_1*n_2)$ which would be the case with the equivalent \verb|\multiput|
statements. This command can be useful in making pictures where a pattern is
repeated at regular intervals in two dimensions, such as certain kinds
of transition diagrams. An illustration of the \verb|\matrixput| command is
presented below.

\begin{center}
\unitlength = 1mm
\begin{picture}(155,35)(0,-8)
\thicklines
\matrixput(0,0)(10,0){6}(0,10){3}{\circle{4}}
\matrixput(2,0)(10,0){5}(0,10){3}{\line(1,0){6}}
\matrixput(0,2)(10,0){6}(0,10){2}{\line(0,1){6}}
\put(60,10){\makebox(0,0)[l]{\tt \shortstack[l]{%
\bs matrixput(0,0)(10,0)\lb 6\rb (0,10)\lb 3\rb\lb\bs circle\lb
4\rb\rb\\[3mm]
\bs matrixput(2,0)(10,0)\lb 5\rb (0,10)\lb 3\rb\lb\bs line(1,0)\lb
6\rb\rb\\[3mm]
\bs matrixput(0,2)(10,0)\lb 6\rb (0,10)\lb 2\rb\lb\bs line(0,1)\lb 6\rb\rb
}}}
\end{picture}
\end{center}

Note: The \verb|\matrixput| command does not restrict the $\Delta x$'s and
the $\Delta y$'s to be zero. The {\it matrix} of {\it objects} can be
``skewed'', i.e., with nonzero $\Delta x$'s and/or $\Delta y$'s.

\subsection{{\tt \bs}grid}

{\tt SYNOPSIS:\\
\hspace*{\leftmargin}%
\bs grid({\it width},{\it height})($\Delta width$,$\Delta height$)%
[{\it initial-X-integer},{\it initial-Y-integer}]}

For example, the following are all valid commands:

\begin{quote}
\begin{verbatim}
\put(0,0){\grid(95,100)(9.5,10)}
\put(0,0){\grid(100,100)(10,5)[-10,0]}
\put(0,0){\tiny \grid(100,100)(5,5)[0,0]} % the numbers in \tiny font.
\put(50,50){\makebox(0,0){\tiny \grid(20,20)(4,4)}}
\end{verbatim}
\end{quote}

The \verb|\grid| command makes a grid of size {\it width\/} units by {\it
height\/} units where vertical lines are drawn at intervals of $\Delta width$
and horizontal lines at intervals of $\Delta height$. The major motivation
for this command is that making a grid in the picture initially can be
very useful when laying out pictures -- it's like having a graph underneath
the picture which can be eventually deleted or commented out. Moreover, one
might actually want a grid as an object in its own right! Figure~1 (on page
14) presents an example of this command.

The {\it width\/} and {\it height\/} should be divisible by their respective
$\Delta$'s, otherwise the grid will not be of correct dimensions. The numbers
in \verb|[ ]| at the end are optional. Their absence makes a simple grid with
lines. Their presence makes a ``numbered'' grid with integers around the
borders where the numbers put have the starting value as specified in
\verb|[. , .]| argument and are incremented by $\Delta width$ and $\Delta
height$ respectively. If specified, then these starting numbers must be
integers. The dimensions are all in units and do not have to be integers,
although in most cases one will want integers only. There is an additional
constraint when plotting a ``numbered'' grid --- the
\mbox{``$\Delta$''-dimensions} have to be integers, since one cannot easily
generate real numbers from within \TeX. None of the errors of this kind are
caught, hence, if the grid comes out funny, one of the above-mentioned
conditions may have been violated.

The \verb|\grid| command produces a box and thus needs to be \verb|\put| at
the required coordinates. The reference point of the grid is the bottom-left
corner and the numbers along the borders, if any, do not affect the reference
point. If it is desired to have another reference point, then the whole grid
statement may be put in a \verb|\makebox(0,0)[..]{...\grid...}| kind of
construction.

\subsection{{\tt \bs}dottedline}

{\tt SYNOPSIS:\\
\hspace*{\leftmargin}%
\bs dottedline[{\it optional dotcharacter}]\lb {\it dotgap in units}\rb%
($x_1$,$y_1$)($x_2$,$y_2$)\ldots($x_n$,$y_n$)}

The above command connects the specified points by drawing a dottedline
between each pair of coordinates. At least two points must be
specified. The dotted line is drawn with inter-dot gap as specified in the
second argument (in unitlengths). Note that since integral number of dots
have to be plotted, the interdot-gap may not necessarily be exactly as
specified, but very close. It really doesn't matter in visual appearance
except when the length of dottedline is very small. By default, a little
square (\verb|\picsquare|, described later) is used as the dot, and can be
changed by optionally specifying another character. The thickness of dots is
governed by currently effective \verb|\thinlines|, \verb|\thicklines| or
\verb|\linethickness...| declaration when the default character is used. Note
that some characters such as ``*'' in roman font do not come out centered,
although most other characters do.

One can obtain a {\it solid line\/} by specifying a very small inter-dot gap.
Since \LaTeX\ provides for only finite number of slopes for drawing lines,
this gives a general way of making lines with arbitrary slopes. However, if
{\it solid lines\/} are made using above technique, there is a good chance
\TeX\ will run out of memory, hence it is suggested that this command be used
only for ``dotted'' lines. Another, much more efficient, way of making solid
lines is described later in the section on \verb|\drawline|.

Each ``dot'' in the dottedline is plotted as a centered object, including
those at the end points. Thus, a dottedline with a large-sized {\it
dotcharacter\/} may appear to be longer although, technically speaking,
correct. To clarify the point, below are three lines of equal length and, in
the case of dottedlines, with equal spacing:

\begin{center}
\begin{picture}(150,20)(0,-5)
\put(0,10){\line(1,0){70}}
\put(0,-4){\line(0,1){18}}
\put(70,-4){\line(0,1){18}}
\thicklines
\dottedline{3}(0,5)(70,5)
\dottedline[$\bullet$]{3}(0,0)(70,0)
\put(80,9){\tt \bs put(0,10)\lb \bs line(1,0)\lb 70\rb \rb }
\put(80,4){\tt \bs dottedline\lb 3\rb (0,5)(70,5)}
\put(80,-1){\tt \bs dottedline[\$\bs bullet\$]\lb 3\rb (0,0)(70,0)}
\end{picture}
\end{center}

\subsection{{\tt \bs}dashline}

{\tt SYNOPSIS:\\
\hspace*{\leftmargin}%
\bs dashline[{\it stretch}]\lb{\it dash-length}\rb
[{\it inter-dot-gap for dash}]%
($x_1$,$y_1$)($x_2$,$y_2$)\ldots($x_n$,$y_n$)}\newline
where {\it stretch} is an integer between -100 and infinity.

The above command connects the specified points by drawing a dashline
between each pair of coordinates. At least two points must be
specified. A \verb|\dashline| is a dashed line where each {\it dash\/} is
constructed using a {\it dottedline\/}%
\footnote{for efficiency, in the case of horizontal and vertical dashlines,
the dash is constructed using a rule.}.
The dash-length is the length of the {\it dash\/} and inter-dot-gap is the gap
between each dot that is used to construct the dash, both in unitlengths.

By default, a solid looking dash is constructed, but by specifying an
inter-dot-gap in the third argument, different looking dashes may be
constructed. With a large inter-dot-gap (about $>$0.4mm), each dash will have
the appearance of a little dotted line. One can create a variety of {\it
dashlines} where each {\it dash\/} looks different. Here are a few sample
dashlines:

\begin{center}
\begin{picture}(150,30)(0,-5)
\dashline{4}[0.7](0,18)(60,18)
\thicklines
\dashline{4}(0,11)(60,11)
\dashline[-30]{4}(0,7)(60,7)
\put(75,20){\makebox(0,0)[tl]{\tt \shortstack[l]{%
\bs dashline\lb 4\rb [0.7](0,18)(60,18)\\
\bs thicklines \\
\bs dashline\lb 4\rb (0,11)(60,11)\\
\bs dashline[-30]\lb 4\rb(0,7)(60,7)
}}}
\end{picture}
\end{center}

The {\it stretch\/} in \verb|[ ]| is an integer percentage and implies
a certain ``stretch'' for positive values and ``shrink'' for negative
values; it is optional and by default is ``0'' unless the default
itself has been changed (described later). The number ``0'' signifies
that a minimum number of dashes be put such that they are
approximately equally spaced with the empty spaces between them. A
$+$ve number means increase the number of dashes by {\it stretch}
percent, and a $-$ve number means reduce by that percent. By reducing
the number of dashes, the empty space between dashes is stretched
while maintaining the symmetry. The lower limit on {\it stretch\/} is
obviously -100 since at less than -100\% reduction one essentially
gets nothing. On the upper side, the number, theoretically, can be as
large as infinity (barring arithmetic overflows) and the macro does
not check for any upper bound; one should normally not require more
than 100 percent increase (100 $\Rightarrow$ double the number of
dashes) since that would essentially mean a ``solid line'' and it is
more efficient to use the \verb|\drawline| command for drawing such
lines, as described later.

The idea behind the {\it stretch\/} percentage option is that if several
dashed lines of different lengths are being drawn, then all the dashed lines
with the same $-$ve or $+$ve {\it stretch\/} will have similar visual
appearance, as might be desired if one were plotting a graph --- one would
like a particular ``curve'' to look the same between all the points on that
curve. Also, it can be used to take any corrective actions, if the appearance
of the default dashline does not meet one's approval.

The default {\it stretch\/} percentage can be changed by a
\verb|\renewcommand| on the parameter\linebreak
\verb|\dashlinestretch| any time and it takes effect immediately. The
argument is the integer percentage increase or reduction that will be
applied to all \verb|\dashline| commands except the ones in which the
percentage is explicitly given using \verb|[ ]| optional parameter.
For example, all {\it dashlines\/} could be reduced by 50 percent by
putting the following line {\it before\/} using any \verb|\dashline|
command:\\[2mm]
\hspace*{\leftmargin}\verb|\renewcommand{\dashlinestretch}{-50} % ONLY
INTEGERS PERMITTED.|

An explicit argument to the \verb|\dashline| command in \verb|[ ]|
overrides any default values, so for instance, after the above declaration,
if a dashline with ``0'' stretch was desired,
then one would simply say:\\[2mm]
\hspace*{\leftmargin}\verb|\dashline[0]{...}(x1,y1)(x2,y2)  
% where "0" implies no stretch or shrink|

A note about dashlines of small length. All dashlines always have a dash
beginning at the first coordinate and another ending at the second
coordinate, which implies that a minimum of two dashes are plotted. For small
lines (or larger lines with accordingly larger sized dashes) the dash-length
is reduced as much as necessary to meet above conditions; in such cases, if
necessary, the $-$ve stretch arguments are ignored. Such dashlines
usually do not have an acceptable appearance, and may either be omitted or be
plotted separately as a dottedline or a dashline with a small dash-length.

\subsection{{\tt \bs}drawline}

{\tt SYNOPSIS:\\
\hspace*{\leftmargin}%
\bs drawline[{\it stretch}]($x_1$,$y_1$)($x_2$,$y_2$)\ldots($x_n$,$y_n$)}\\
where {\it stretch\/} is an integer between -100 and infinity.

The above command connects the specified points by drawing a line between
each pair of coordinates using line segments of the closest slope available
in the fonts. At the minimum two points must be specified. Since there are
only finite number of slopes available in the line segment fonts, some lines
appear jagged. A \verb|\drawline| can be thick or thin depending on the
\verb|\thinlines| or \verb|\thicklines| declaration in effect; these are the
only two thicknesses available for such lines. This is also the most
efficient, in terms of memory and cpu usage, way of drawing lines of
arbitrary slopes.

The {\it stretch\/} parameter has properties similar to those described
earlier in the context of dashlines. It is again a percentage and implies a
certain ``stretch'' or ``shrink''; it is optional and by default is ``0''
unless the default itself has been changed (described later). The same rules
apply to the range of the {\it stretch\/} value. In this case, the number
``0'' signifies that a minimum number of dashes be put such that the line
appears solid and each dash ``connected'' at the ends. By reducing the number
of dashes by specifying a $-$ve {\it stretch\/}, one effectively gets a
dashed line. On the other hand, by specifying a $+$ve {\it stretch\/}, more
dashes will be used in constructing the line, giving a less jagged
appearance.

A parameter, namely, \verb|\drawlinestretch|, has been provided for
\verb|\drawline|'s and its usage is identical to \verb|\dashlinestretch|
described earlier in the context of \verb|\dashline|.

A limitation of drawing lines using line-segment fonts is that the length of
segments is fixed and is not user-controllable. If explicit control over the
line-segment length is desired, then \verb|\dashline| may be used.
If the length of the line to be drawn is smaller than the length of available
line segment, then a solid line is constructed using \verb|\dottedline| with
dots being very close; the thickness of the {\it line\/} thus constructed is
chosen appropriately. Note that in such a case, only a solid line can be
constructed between the two points, i.e., dashed appearance can not
be given to such small lines, and any $-$ve stretch is ignored.

\subsection{The join environments}
\begin{tt}
SYNOPSIS:
\begin{quote}
\bs jput($x$,$y$)\lb{\it object}\rb

\bs begin\lb dottedjoin\rb[{\it optional dotcharacter}]%
\lb{\it inter-dot-gap}\rb\\
.....\hspace*{1in} dottedlines drawn here for each \verb|\jput| statement.\\
\bs end\lb dottedjoin\rb

\bs begin\lb dashjoin\rb[{\it stretch}]%
\lb{\it dash-length}\rb[{\it inter-dot-gap for dash}]\\
.....\hspace*{1in} dashlines drawn here for each \verb|\jput| statement.\\
\bs end\lb dashjoin\rb

\bs begin\lb drawjoin\rb[{\it stretch}]\\
.....\hspace*{1in} drawlines drawn here for each \verb|\jput| statement.\\
\bs end\lb drawjoin\rb

\end{quote}
\end{tt}

Three environments, corresponding to the three kinds of lines described
earlier, are also provided. They are {\tt dottedjoin}, {\tt dashjoin} and
{\tt drawjoin}. All the three environments use yet
another new command \verb|\jput|\footnotemark\ (join and put) which is
identical to the regular \verb|\put| command of \LaTeX\ except that it
behaves differently when in any of the three environments.

\footnotetext{could have redefined the {\tt \bs put} statement; {\tt \bs
jput} behaves identically to {\tt \bs put} when not in any join
environment.} All {\it objects\/} put using a \verb|\jput| command within the
scope of any of the three environments are, in addition to being plotted,
joined by lines of the respective kind; in other words, a line of the
specified kind is drawn between {\it points\/} plotted using \verb|\jput|
statement in the order they are encountered; a {\it point\/} refers to the $x$
and $y$ coordinates specified in the \verb|\jput| statement. Consecutive
\verb|\jput| statements are assumed to define adjacent points --- hence, the
input should be accordingly ordered. Moreover, the plotted point should be in
a \verb|\makebox(0,0){...}| (except, of course, centered {\it objects\/} such
as \verb|\circle| and \verb|\circle*|) if it is to be centered on the
specified coordinate; without it the object's bottom-left corner will be at
the specified coordinate. Each instance of any of the three join
environments defines a separate ``curve'' hence every set of points belonging
to different ``curves'' should be enclosed in separate join environments.

All the parameters, optional and mandatory, other than the coordinates that
go along with the line drawing commands, may be specified after the
\verb|\begin{...join}| command as its arguments. Currently effective default
values are used when not specified in {\tt []}, and may be changed anytime
using the \verb|\renewcommand| as discussed previously.

The primary motivation for designing the join environments is for use in
plotting graphs and joining different curves by different looking lines.
It is not necessary that the \verb|\jput| statements put some object; if
the object is null then one gets only lines --- in such a case it is much
simpler to use the respective line drawing command directly.

\subsection{{\tt \bs}picsquare}

\verb|\picsquare| is a simple macro that gives a little square dot with its
center as the reference point. The size of the square is dependent on the
currently effective \verb|\thinlines|, \verb|\thicklines| or
\verb|\linethickness...| declaration. Most of the commands described earlier
that plot little dots, use this macro\footnotemark. It has been provided
primarily to be used in conjunction with \verb|\putfile| command described
below.
\footnotetext{The {\tt \bs dottedline} macro actually uses another similar
macro {\tt \bs picsquare@bl}, which gives an identical square, but with the
bottom-left corner as the reference point.}
Only {\tt \bs picsquare} has been made accessible to the user.

\subsection{{\tt \bs}putfile}

{\tt \bs putfile\lb{\it filename}\rb\lb{\it object}\rb}

The command \verb|\putfile| is similar to the \verb|\put| command except that
the $x$ and $y$ coordinates required by the \verb|\put| command are read from
an external file and the same {\it object} is plotted at each of those
coordinates.

The motivation behind this command is that \TeX\ does not have the capability
to do floating point calculations which would be required if one wished
to plot any parametric curve other than straight lines. Coordinates for such
curves can be easily generated by programs in other languages and
subsequently a ``dotted'' curve can be plotted via \TeX\ or \LaTeX. Even if
coordinates for certain curves could be generated from within \TeX, it is
much more efficient to use other languages --- eventually only the
coordinates of the points are required.
For instance, one can use the Unix%
\footnote{{\it Unix} is a trademark of AT\&T.}\ facility
{\it spline\/} to generate smooth curves with equidistant ``dots''.

{\bf Format of the External File:} The external file of coordinates must have
``$x\;y\,$'' pairs, one pair on each line, with a space between them. Also, it
is suggested that some extension such as ``{\tt .put}'' be used for such
data files to distinguish them from regular text files in which case it must
be explicitly specified in the first argument so that \TeX\ doesn't look for
a ``{\tt .tex}'' extension.

The ``{\tt \%}'' character remains valid as a comment character and such lines
are ignored. However, there should be at least one space after the second
entry if a comment is on the same line as data since {\tt \%} eats up the
newline.

For example, to plot a smooth curve along a set of coordinates, one may
undertake the following steps:
\begin{enumerate}
\item have a file of ``$x\;y$'' coordinates for original data points, say,
{\tt datafile}.
\item run the command (for Unix systems): {\tt spline -200 datafile $>$
data.put}
\item in a picture environment in a \LaTeX\ file, put the command:\\
\hspace*{20pt}\verb|\putfile{data.put}{\picsquare}|\\
(see previous section for explanation of \verb|\picsquare|).
\end{enumerate}

\section{General Comments}

A few remarks about efficiency and quirks:
\begin{itemize}
\item In most of the above commands, simply typing a {\tt []} for optional
arguments with {\it nothing\/} as the value will either cause an error or will
be interpreted as a null value; hence a {\tt []} should not be typed if an
optional argument is not meant to be specified.

\item If too many ``dots'' are to be plotted in one picture, it is suggested
that a character other than the default be used --- about 40--50\% more dots
can be plotted in a picture using a period (.) or a \verb|\bullet|
($\bullet$) in various sizes, rather than the default \verb|\picsquare|,
although the latter seems to have a better visual appearance. The
use a \verb|\picsquare| also enables one to have a better control over the
thickness of dots and lines.

A note on efficiency: when specifying a font or a fontsize for a character
it is more efficient to say:\\
\verb|{\tiny \dottedline[$\bullet$]{2}(0,0)(40,30)(80,10)}|,\ \ \ rather than\\
\verb|\dottedline[\tiny $\bullet$]{2}(0,0)(40,30)(80,10)|.\\
In the latter case, \verb|\tiny| macro gets invoked for {\it each\/} instance
of the dotcharacter \verb|$\bullet$| as the dottedline is plotted.

\item If it is not very important as to how accurately spaced a dashed line
appears, then it is suggested that \verb|\drawline| command with a $-$ve
stretch be used instead of \verb|\dashline|, since the former is much more
cpu- and memory-efficient.

\item \verb|\dottedline| and \verb|\dashline| come out much too thin with
\verb|\thinlines|. Moreover, the thicker the \verb|\dashline|, fewer
``dots'' are required to construct dashes resulting in lesser memory and cpu
usage. Thus, it is recommended that they be plotted with \verb|\thicklines|
in effect, or with a linethickness of about 1--2pt.

\item In the case of \verb|\drawline|, any explicit linethickness declarations
(i.e. using \verb|\linethickness| command) are ignored. The only applicable
declarations are \verb|\thinlines| and \verb|\thicklines| since line-segment
fonts are available in only two thicknesses.
\end{itemize}

Above commands are available in the picture environment only since they use
many of the \LaTeX 's predefined picture commands. Extensive use of some of
the internal macros and variables of \LaTeX\ has been made for efficiency
sake, even though that makes these macros vulnerable to future revisions of
\LaTeX.

The dottedline macro gets complicated because \TeX\ does not have any builtin
facility for floating point calculations or for calculating square-roots or
trigonometric functions. The inter-dot-gap in a dottedline has to be treated
as the actual distance between two dots along the ``hypotenuse'' and not its
projected distance along x-axis or y-axis, since the latter interpretation
would result in a different {\it real\/} inter-dot-gap for different slopes;
it would be incorrect if we were joining points on a graph. The dootedline
macro treats the inter-dot-gap as the actual distance between two dots and
draws the various segments of the ``curve'' with this distance fixed. The
macro accomplishes this by estimating the actual length of the line and the
number of segments of the specified distance that will fit between the two
end-points; a macro, namely, \verb|\sqrtandstuff| calculates this
square-root. Some algebraic relations are used in estimating this square-root
and are described in appendix A.

Beware, if far too many dots are put in one picture, \LaTeX\ will run out of
memory (box full), so be kind to it. For instance, by reducing the
inter-dot-gap to about 0.3mm in the case of a \verb|\dottedline|, one can get
essentially a solid line, but that would mean a LOT of dots and it may run
out of memory.

If many lines using above-mentioned macros are drawn, then a
\verb|\clearpage| ought to be put at judicious places in the document so as
to tell \LaTeX\ not to keep those figures floating -- \LaTeX\ sometimes
keeps entire figures in memory while trying to figure out how and where to
lay them and it can frequently run out of memory. A \verb|\clearpage| may
prevent running out of memory and may reduce execution times.
In case of such a
memory-full error message, a \verb|\clearpage| in the region where the error
occurred should be attempted first and if that does not help then the number
of ``dots'' in the picture will have to be reduced.

A word about \verb|\drawline| is in order. \LaTeX 's \verb|\line| command
takes an ordered pair of integers to specify the slope of the line where the
numbers are between $-$6 and 6 such that the least common divisor is 1. For
the \verb|\drawline| command, the given arbitrary slope has to be mapped to
the pair of integers representing the {\it closest\/} available slope.
Another macro, \verb|\lineslope| is used to accomplish this task. The macro
\verb|\lineslope| takes two arguments, the base and the height of the
triangle whose hypotenuse represents the line to be drawn and returns the
ordered pair of integers representing the closest slope; using a line segment
of that slope, a jagged line between the two specified end-points is then
constructed. More details can be found in the macro file {\tt epic.sty}.

As noted earlier, the command \verb|\jput| behaves identically as \verb|\put|
when not in any of the join environments. The author considered obliterating
the \verb|\put| command too radical a step. Also, there should have been a
command \verb|\jputfile| corresponding to the \verb|\jput| command (like the
\verb|\putfile| command) but that was considered unnecessary since typically
the number of coordinates plotted in a join environment would be an order
less than what might be the case with \verb|\putfile| and can be easily
typed explicitly in the document using \verb|\jput| commands.
However, if it is desired to have all the \verb|\put| commands treated as
though they were \verb|\jput|, the following declaration may be used:\\
\hspace*{\leftmargin}\verb|\let\put\jput| \\
Above declaration will make all
the \verb|\put| commands be treated as \verb|\jput|; in particular,
\verb|\putfile| command would then behave as though it were a
\verb|\jputfile| when in any of the join environments. However, it is
suggested that such ``tricks'' be used with care.

Finally, commands to plot vectors of arbitrary slopes have not been
implemented. One way to plot them is to plot a line, and subsequently plot a
\verb|\vector| of appropriate slopes and length zero at the required place.

Following pages contain some examples. The test-sample picture for
\verb|\drawline| command (Figure~2) is also about the maximum amount of
objects that one can put in one picture. Older versions of \TeX\ and \LaTeX\
may not be able to print pictures of this size.

\newpage
%% FIGURES FOLLOW
\begin{figure}[t]
\begin{center}
%\newcommand{\plotchar}{\makebox(0,0){\large $\otimes$}}
\unitlength = 1mm
\begin{picture}(100,100)(0,0)
\put(0,0){\tiny \grid(100,100)(5,5)[0,0]}
\drawline(10,5)(60,10)(85,20)(90,60)(100,95)
\drawline[-50](10,0)(65,5)(90,15)(95,55)
\thicklines
\dottedline{1.4}(10,10)(60,20)(75,35)(95,95)
\dashline{2}(80,90)(50,80)(30,50)(10,40)
\dashline{2}[0.5](80,80)(50,70)(30,40)(10,30)
\dashline[-30]{2}[0.5](80,70)(50,60)(30,30)(10,20)
\end{picture}

\end{center}
\caption[]{\normalsize An Example of Various Line Drawing Commands}
\end{figure}

\begin{small}
\verbfile{picman-fig1}
\end{small}

\newcount\xjunk
\newcount\yjunk

\clearpage
\begin{figure}
\begin{center}
\begin{tiny}
\begin{picture}(140,140)(-70,-70)
\thinlines
\xjunk=60 \yjunk=3
\loop
\drawline(0,0)(\xjunk,\yjunk)
\drawline(0,0)(-\xjunk,\yjunk)
\drawline(0,0)(\xjunk,-\yjunk)
\drawline(0,0)(-\xjunk,-\yjunk)
\put(\xjunk,\yjunk){\plotchar}
\put(-\xjunk,\yjunk){\plotchar}
\put(\xjunk,-\yjunk){\plotchar}
\put(-\xjunk,-\yjunk){\plotchar}
\put(\xjunk,\yjunk){\makebox(8,0)[l]{\ \ (\number\xjunk,\number\yjunk)}}
\put(-\xjunk,\yjunk){\makebox(-4,0)[r]{(\number-\xjunk,\number\yjunk)}}
\put(\xjunk,-\yjunk){\makebox(8,0)[l]{\ \ (\number\xjunk,\number-\yjunk)}}
\put(-\xjunk,-\yjunk){\makebox(-4,0)[r]{(\number-\xjunk,\number-\yjunk)}}
\ifnum\yjunk < 53 \advance\yjunk by 10 %will go till 53.
\repeat
\xjunk=3 \yjunk=60
\loop
\drawline(0,0)(\xjunk,\yjunk)
\drawline(0,0)(-\xjunk,\yjunk)
\drawline(0,0)(\xjunk,-\yjunk)
\drawline(0,0)(-\xjunk,-\yjunk)
\put(\xjunk,\yjunk){\plotchar}
\put(-\xjunk,\yjunk){\plotchar}
\put(\xjunk,-\yjunk){\plotchar}
\put(-\xjunk,-\yjunk){\plotchar}
\put(\xjunk,\yjunk){\makebox(0,7)[t]{(\number\xjunk,\number\yjunk)}}
\put(-\xjunk,\yjunk){\makebox(0,5)[t]{(\number-\xjunk,\number\yjunk)}}
\put(\xjunk,-\yjunk){\makebox(0,-4)[t]{(\number\xjunk,\number-\yjunk)}}
\put(-\xjunk,-\yjunk){\makebox(0,-6)[t]{(\number-\xjunk,\number-\yjunk)}}
\ifnum\xjunk < 53 \advance\xjunk by 10
\repeat

\end{picture}
\end{tiny}
\end{center}
\caption[]{\normalsize Test Sample: Lines of various slopes with
{\tt thinlines}}
\end{figure}

\clearpage
\begin{figure}
\begin{center}
\begin{tiny}
\begin{picture}(140,140)(-70,-70)
\renewcommand{\drawlinestretch}{-50}
\thicklines
\xjunk=60 \yjunk=3
\loop
\drawline(0,0)(\xjunk,\yjunk)
\drawline(0,0)(-\xjunk,\yjunk)
\drawline(0,0)(\xjunk,-\yjunk)
\drawline(0,0)(-\xjunk,-\yjunk)
\put(\xjunk,\yjunk){\plotchar}
\put(-\xjunk,\yjunk){\plotchar}
\put(\xjunk,-\yjunk){\plotchar}
\put(-\xjunk,-\yjunk){\plotchar}
\put(\xjunk,\yjunk){\makebox(8,0)[l]{\ \ (\number\xjunk,\number\yjunk)}}
\put(-\xjunk,\yjunk){\makebox(-4,0)[r]{(\number-\xjunk,\number\yjunk)}}
\put(\xjunk,-\yjunk){\makebox(8,0)[l]{\ \ (\number\xjunk,\number-\yjunk)}}
\put(-\xjunk,-\yjunk){\makebox(-4,0)[r]{(\number-\xjunk,\number-\yjunk)}}
\ifnum\yjunk < 53 \advance\yjunk by 10 %will go till 53.
\repeat
\xjunk=3 \yjunk=60
\loop
\drawline(0,0)(\xjunk,\yjunk)
\drawline(0,0)(-\xjunk,\yjunk)
\drawline(0,0)(\xjunk,-\yjunk)
\drawline(0,0)(-\xjunk,-\yjunk)
\put(\xjunk,\yjunk){\plotchar}
\put(-\xjunk,\yjunk){\plotchar}
\put(\xjunk,-\yjunk){\plotchar}
\put(-\xjunk,-\yjunk){\plotchar}
\put(\xjunk,\yjunk){\makebox(0,7)[t]{(\number\xjunk,\number\yjunk)}}
\put(-\xjunk,\yjunk){\makebox(0,5)[t]{(\number-\xjunk,\number\yjunk)}}
\put(\xjunk,-\yjunk){\makebox(0,-4)[t]{(\number\xjunk,\number-\yjunk)}}
\put(-\xjunk,-\yjunk){\makebox(0,-6)[t]{(\number-\xjunk,\number-\yjunk)}}
\ifnum\xjunk < 53 \advance\xjunk by 10
\repeat

\end{picture}
\end{tiny}
\end{center}
\caption[]{\normalsize \parbox[t]{110mm}{Test Sample: Dashed lines of various
slopes using {\tt \bs drawline} command with {\tt linethickness=1pt} and {\tt 
drawlinestretch = -50}}}

\end{figure}

\clearpage
\renewcommand{\baselinestretch}{0.85}
\begin{footnotesize}
\verbfile{picman-fig3}
\end{footnotesize}

\clearpage
%Following commands were used to produce the graph on the next page.

\newcommand{\plotcharms}{\makebox(0,0){$\otimes$}}
\newcommand{\plotcharscs}{\circle{1.5}}
\newcommand{\plotcharcs}{\makebox(0,0){$\diamond$}}
\begin{figure}
\begin{center}
\begin{picture}(140,160)(-130,-10)
\linethickness{0.4mm}
\put(-130,0){\vector(1,0){140}}
\put(-130,0){\vector(0,1){150}}
\thicklines
\multiput(0,-1)(-10,0){14}{\line(0,1){2}}
\multiput(-131,0)(0,10){15}{\line(1,0){2}}
\multiputlist(0,-4)(-20,0){0,50,100,150,200,250,300} %numbers along X-axis
\multiputlist(-132,20)(0,20)[r]{10,20,30,40,50,60,70} %numbers along Y-axis
\put(-60,-10){\makebox(0,0){Interarrival Times (msec.)}}
\put(-141,75){\makebox(0,0){\shortstack{%
N\\o\\r\\m\\a\\l\\i\\z\\e\\d\\[3ex]L\\i\\f\\e\\t\\i\\m\\e\\s}}}
\thinlines
\put(-120,150){\makebox(0,0)[tl]{\fbox{\shortstack[l]{
 {\makebox(4,2)[lb]{\put(2,1){\plotcharms}}}: Message Switching\\[0.5mm]
 {\makebox(4,3)[lb]{\put(2,1){\plotcharscs}}}: Staged Circuit Switching\\[0.5mm]
 {\makebox(4,3)[lb]{\put(2,1){\plotcharcs}}}: Circuit Switching\\[0.5mm]
 {\makebox(2,0)[b]{}}
 }}}}
%
\begin{dottedjoin}{2}
\thicklines
\jput(-120.00000,  34.44896){\plotcharms}
\jput( -60.00000,  35.55244){\plotcharms}
\jput( -40.00000,  36.57292){\plotcharms}
\jput( -30.00000,  37.71716){\plotcharms}
\jput( -20.00000,  40.15218){\plotcharms}
\jput( -12.00000,  48.16034){\plotcharms}
\jput(  -8.00000,  67.75840){\plotcharms}
\jput(  -7.60000,  74.27934){\plotcharms}
\jput(  -7.20000,  83.02326){\plotcharms}
\end{dottedjoin}
%
\begin{dashjoin}{2}
\jput(-120.00000,  15.01202){\plotcharscs}
\jput( -60.00000,  15.95818){\plotcharscs}
\jput( -40.00000,  17.15990){\plotcharscs}
\jput( -30.00000,  18.16152){\plotcharscs}
\jput( -20.00000,  20.32388){\plotcharscs}
\jput( -12.00000,  27.05212){\plotcharscs}
\jput(  -8.00000,  41.58512){\plotcharscs}
\jput(  -7.60000,  45.3435){\plotcharscs}
\jput(  -7.20000,  51.52414){\plotcharscs}
\end{dashjoin}
%
\begin{drawjoin}
\jput(-120.00000,  15.17960){\plotcharcs}
\jput( -80.00000,  16.71960){\plotcharcs}
\jput( -60.00000,  18.29430){\plotcharcs}
\jput( -56.00000,  19.81980){\plotcharcs}
\jput( -52.00000,  20.31963){\plotcharcs}
\jput( -48.00000,  50.24912){\plotcharcs}
\jput( -44.00000,  56.96844){\plotcharcs}
\end{drawjoin}
\end{picture}
\end{center}
\caption[]{A real-life example of a graph}
\end{figure}


\clearpage
\section{Installation and Usage of the Package}
This package of new commands for the picture environment has been implemented
as a documentstyle option ``{\tt epic}''. To include these commands, ``epic''
should be added as an option in the \verb|\documentstyle| command,
e.g.:\\[1mm]
\hspace*{\leftmargin}\verb|\documentstyle[epic]{article}|

For the above option to work, one of the following will have to be done prior
to its use:
\begin{enumerate}
\item A copy of the macro file {\tt epic.sty} be put in the standard place for
such macros (typically {\tt /usr/lib/tex/macros}), or

\item A copy of {\tt epic.sty} be put in some other directory, and the path
declared in the environment variable TEXINPUTS; e.g.\ for C-shell on unix
systems, put a command similar to the following in the ``.cshrc'' file:\\[1mm]
\hspace*{20pt}{\tt setenv TEXINPUTS .:/usr/lib/tex/macros:/users/podar/texlib}
\end{enumerate}

Above environment variable is the directory search path for files specified
in an \verb|\input| or an \verb|\openin| command.

\section{Concluding Remarks}
The implementation of the new commands for the picture environment has been
done with the \LaTeX\ version 2.09 and \TeX\ version 2. They have also been
tested to work with \LaTeX\ version 2.08. These commands may not work with
earlier versions of \TeX\ and \LaTeX.

Most of the commands have been tested fairly thoroughly.
No major revisions are anticipated in the near
future, except, of course, bug fixes. The author welcomes any comments,
constructive or otherwise, suggestions for improvements, any ideas for
possible future revisions and, of course, bugs. It is also requested that he
be informed of any significant changes or modifications made to these
macros.

All the help and encouragement from colleagues in the Dept. of Computer
Science at SUNY at Stony Brook is gratefully acknowledged; in particular,
Soumitra Sengupta's and Divyakant Agrawal's criticisms (often constructive),
help with proofreading the numerous versions of this report and general moral
support were critical to the completion of this project and are thankfully
acknowledged.

\bigskip\noindent%
\begin{small}%
Author's address:\\[3mm]
\begin{tabular}{@{}ll}
USMAIL:&Dept. of Computer Science, SUNY at Stony Brook, Stony Brook,
N.Y.\ 11794\\
CSNET: &podar@sbcs.csnet\\
ARPA: &podar\%suny-sb.csnet@csnet-relay.arpa\\
UUCP: &\{allegra, hocsd, philabs, ogcvax\}!sbcs!podar
\end{tabular}
\end{small}

\begin{thebibliography}{99}
\bibitem{Knu} D.\ E.\ Knuth, ``The \TeX book'', Addison-Wesley Publishing Co.,
1984.
\bibitem{Lam} L.\ Lamport, ``\LaTeX: A Document Preparation System'',
Addison-Wesley Publishing Co., 1986.
\end{thebibliography}

\appendix
%\section{Estimating Pythagorean Square-root}
\medskip\noindent
{\Large\bf Appendix A\ \ \ Estimating Pythagorean Square-root}

\medskip\noindent
For the line drawing commands described in the main sections of this
document, we need to estimate the Pythagorean square-root in order to
determine the length of the line (along its slope). More precisely, we need
to estimate the number of segments of a given length needed to draw a line.
\TeX\ does not provide for floating point calculations, and thus there are no
direct means of calculating the above square-root. Most standard numerical
techniques are iterative and would be too slow when used with \TeX\ for lack
of floating point calculations, and in particular, real division, since
calculation of such a square-root is needed very frequently.

A simple non-iterative formula for estimating the square-root is derived and
described below.

\bigskip\noindent
{\bf Problem: } Given $a$ and $b$, to find $c$ = $\sqrt{a^2 + b^2}$ using
only operations in \{$+,-,*,/$\}.

We can get very tight bounds on the square-root as follows.
Without loss of generality, let $a \ge b$. We seek a simple $n$
such that:
\[\sqrt{a^2 + b^2} \ge a + \frac{b}{n}\]

Squaring both sides, we have
\[\begin{array}{lrcl}
\Leftrightarrow & a^2+b^2 &\ge& a^2 + \displaystyle\frac{b^2}{n^2} +
\displaystyle\frac{2ab}{n} \\[2mm]
\Leftrightarrow & (1 - \displaystyle\frac{1}{n^2}) b^2 &\ge&
\displaystyle\frac{2ab}{n}\\[2mm]
\Leftrightarrow & \displaystyle\frac{b}{a} &\ge& \displaystyle\frac{2n}{(n^2
-1)}\\[2mm]
\mbox{or }      & (\displaystyle\frac{b}{a})n^2 - 2n -(\displaystyle\frac{b}{a}) &\ge& 0
\end{array}\hfill\]

>From the quadratic equation above, we finally get an expression for $n$,
\[ n \;=\; \frac{2 \pm \sqrt{4 + 4(\frac{b}{a})^2}}{\frac{2b}{a}}
      \;=\; \frac{1 \pm \sqrt{1 + (\frac{b}{a})^2}}{\frac{b}{a}} \]

Only the $+$ve root interests us since $n$ has to be positive.
Note that the term under the root is  bounded above and below (since
$\frac{b}{a} \le 1$):
\[1 \;\le\; \sqrt{1 + (\frac{b}{a})^2} \;\le\; \sqrt{2}\]

Hence, we have two values for $n$,
\[ n_l\;=\; \frac{1+1}{\frac{b}{a}} \;=\; \frac{2a}{b};\;\;\;\;\;\;\;\;\;\;
n_u \;=\; \frac{1+ \sqrt{2}}{\frac{b}{a}} \;=\; \frac{(1+ \sqrt{2})a}{b}
\]
%
which finally gives us a lower and an upper bound for $c$, the Pythagorean
square-root,
\[ a + \frac{b^2}{(1+ \sqrt{2})a} \;\le\; c \;\le\; a + \frac{b^2}{2a}\]

These are very tight bounds. Denoting the lower bound as $c_l$ and upper
one $c_u$, below are some numerical results ($c$ = exact square-root):

\begin{center}
\begin{tabular}{|c|c|c|c|c|}
\hline
a & b & c & $c_l$ & $c_u$\\
\hline
100.0 & 100.0 & 141.4213 & 141.4213 & 150.0\ \ \ \ \ \ \\
100.0 & \ \,80.0  & 128.0642 & 126.5096 & 132.0\ \ \ \ \ \  \\
\ \,30.0  & \ \,20.0  & \ \,36.0555  & \ 35.5228  & \ 36.6667 \\
\hline
\end{tabular}
\end{center}

With the above bounds, one can do a linear interpolation to get exact values.
In our case, since it is not required to be {\it extremely\/} accurate, for 
estimating the square-root in the line drawing commands,
we simply take the midpoint of the two bounds. For small
numbers, which is expected to be the case most of the time,
the error is very small.

With some algebra, we get the mid-point estimate of $c$,
\[c = \frac{c_l+c_u}{2} = a + \frac{b^2 * (3 + \sqrt{2})}{a*4*(1 + \sqrt{2})}
= a + \frac{0.457\: b^2}{a} \;\;\;\;(a \ge b) \]

The macro \verb|\sqrtandstuff| uses the above formula for estimating the
number of points (for \verb|\dottedline| macro) and number of segments (for
\verb|\dashline| macro). The \verb|\sqrtandstuff| macro, instead of
calculating the length of the line, directly calculates the {\it number\/} of
segments of a given length. For example, to draw a dotted line from
($x_1,y_1$) to ($x_2,y_2$) with the inter-dot-gap as $d$, we estimate the
number of dots $n$ using the following expression,
\[ n= \frac{\Delta x}{d} + 
\frac{0.457\:(\frac{\Delta y}{d})^2}{\frac{\Delta x}{d}} \;\;\;\;\;\;\;
\Delta x = |x_2 - x_1| \mbox{ and } \Delta y = |y_2 - y_1|\]
assuming $\Delta x \ge \Delta y$ (otherwise they may be inter-changed).

Note that since divisions in \TeX\ are integer-divisions, it is simpler to
deal in ``number of segments'' rather than actual lengths
(e.g. in the expression above, $\frac{\Delta x}{d} = $ number of segments
along X-axis).

\noindent
{\bf Caveat:} The approach presented here for estimation of
Pythagorean square-root is an independent effort by the author. It may
already exist in the literature --- the author is neither aware of it nor has
he made any serious attempts at uncovering it.


\end{document}
