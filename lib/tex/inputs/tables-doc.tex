\documentstyle[tables]{article}
% Note : This is not the original document TABLEDOC.TEX.  It has been 
%        re-typed using a hardcopy of the original by Atul Kacker
%
\def\bs{\char'134}
\def\vbar{\char'174}
\def\and{\char'046}
\def\tc#1{\hbox{\tt \bs #1}}
\def\ctr#1{\quad #1 \hfil}
\font\sl=cmsl10
\setlength{\textwidth}{433pt}
\setlength{\oddsidemargin}{0pt}
\setlength{\marginparwidth}{72pt}
\setlength{\topmargin}{0pt}
\setlength{\textheight}{555pt}
%
\begin{document}
\centerline{\framebox{\large Making Tables with Macros}}
\vbox{\vskip 0.5in}
\centerline{Ray F. Cowan}
\centerline{22 February 1985}
\vbox{\vskip 0.5in}
Tables have traditionally been difficult to make using \TeX\ -- especially
ruled tables.  The file {\tt \hbox{TABLES.TEX}} contains macros designed to
prepare both ruled and unruled tables with considerably less effort.  Note that
\hbox{\tt TABLES.TEX} can be used with any macro set; it does not depend on
prior loading of \hbox{\tt PHYZZX} or any other macro set, for example. One of
the main advantages of this macro set is that you no longer need to design a
preamble for the table; the macros will scan your table entries and construct a
suitable preamble for you.  To access the macros, say \hbox{\tt `\bs input
TABLES'} in your \TeX\ file, somewhere before you first use them.  The macros
available are listed in Table~\ref{tab:one} and Table~\ref{tab:two}.

\vbox{\vskip 0.5in}
\begin{table}[h]
\begintable
{\sl Macro name} | {\sl Description} \cr
\tc{begintable} | \para{Indicates the start of a new table} \cr
\tc{endtable}   | \para{Ends the current table.  Must be used in
                        the place of the last \tc{cr}.}\cr 
\tc{cr}         | \para{Ends the current row, and starts the next 
                        one.  The completed row will be separated 
                        from the next with a thin horizontal rule.}\cr 
\tc{crthick}    | \para{Similar to \tc{cr}, but the rows will be 
                         separated with a thick horizontal rule.} \cr 
\tc{crnorule} or \tc{nr}
                | \para{Similar to \tc{cr}, but the rows will not be 
                        separated by any rule.}\cr 
{\tt \hbox{\vbar}} (vertical bar) or \tc{vb} 
                | \para{Separates one column from the next, and 
                        puts a vertical rule between them.} \cr 
{\tt \hbox{\and}} or \tc{novb} 
                | \para{Same as {\tt \hbox{\vbar}}, but does not 
                        put in the vertical rule between the columns.} \cr
{\tt \hbox{\vbar}} 
                | \para{Same as {\tt \hbox{\vbar}}, but puts in a 
                        thick vertical rule.} 
\endtable
\caption{Description of simple table macros}\label{tab:one}
\end{table}
\clearpage
\begin{table}[h]
\begintable
\tc{thicksize}={\em dimen}
                | \para{This dimension specifies the thickness of the thick 
                        rules in the table.  The default size is 1.5 
                        points.} \cr
\tc{thinsize}={\em dimen}
                | \para{This dimension specifies the thickness of thin rules
                        in the table.  The default size is 0.8 points.} \cr
\tc{tablewidth}={\em dimen}
                | \para{Specifies how wide to make the next table.  If not
                        specified, the table is made to its natural width.  
                        This value is reset following the construction of
                        each table.} \cr
\tc{multispan}{\tt \{}{\em n}{\tt \}}
                | \para{Makes the next entry span the next {\em n} columns,
                        where {\em n} is an integer, {\em n} $>$ 0.  See other
                        notes on \tc{multispan} below.} \cr
\tc{omit}       | \para{This \TeX\ primitive causes the normal template for
                        its entry to be omitted, allowing the user to do
                        something else with this entry.} \cr
\tc{para}{\tt \{}{\em paragraph text}{\tt \}}
                | \para{Turns an entry into a neat little paragraph like this 
                        one.  The width of the paragraph is determined by the
                        dimension \tc{parasize}.  The default is 4 inches.} \cr
\tc{parasize}={\em dimen}
                | \para{Sets the width of paragraphs produced with the \tc{para}
                        macro.} \cr
\tc{ctr}{\tt \{\#\}} 
                | \para{Used in the standard template, this macro centers 
                        its argument in the column.  The user can redefine it 
                        for special effects.  The default definition is}
                        \crnorule
                | \qquad\tc{def}\tc{ctr}{\tt \#1\{}\tc{hfil}{\tt 
                       \bs\ \#1\bs\ }\tc{hfil}{\tt \}}\cr
\tc{vctr}{\tt \{\#\}}
                | \para{Centers an entry vertically.  The vertical center of the
                        entry is placed on the baseline of the row containing
                        it.  The intended use is to center an entry between two
                        rows.} \cr
\tc{centeredtables}
                | \para{Turns table centering on.  Each table will be centered 
                        left-to-right on the page.  This is the default.} \cr
\tc{noncenteredtables}
                | \para{Turns table centering off. Each table is returned as an
                        \tc{hbox}, and it is up to the user to place it as 
                        desired.} \cr
\tc{tableinfotrue}
                | \para{Turns on the diagnostic message telling you how many
                        rows and columns were found in the table.  This is the
                        default.} \cr
\tc{tableinfofalse}
                | \para{Turns off the diagnostic messages concerning rows and 
                        columns.}
\endtable
\caption{Description of extended table macros}\label{tab:two}
\end{table}
\clearpage
The general idea is that you start your table with the command \tc{begintable},
type your entries in one row at a time, then finish with the command
\tc{endtable}.  To specify a row, enter the individual entries into your \TeX\
file, separating each column entry with a {\tt \hbox{\vbar}}, an {\tt
\hbox{\and}}, or a {\tt \hbox{\bs\vbar}}.  A {\tt \hbox{\vbar}} will separate
the adjoining columns with a thin vertical rule, an {\tt \hbox{\and}} will
leave out the vertical rule, and a {\tt \hbox{\bs\vbar}} will separate the
columns with a thick vertical rule.  To end one row and start another, use
either a \tc{cr}, a \tc{crnorule}, or a \tc{crthick}.  A \tc{cr} separates the
rows with a thin horizontal rule; a \tc{crnorule} leaves out the horizontal
rule, while \tc{crthick} inserts a thick horizontal rule.  Then end the last
row with an \tc{endtable}. 

Each row of the table must contain the same number of columns, otherwise 
unpredictable things will happen.  Again the {\em last row} must {\em not} 
end with \tc{cr...}, but {\em must} end with an \tc{endtable}.  If you put a 
\tc{cr} and an \tc{endtable} both on the last row, you won't like what 
happens.

Each entry will be centered in it's column (unless you use \tc{omit}, a 
\TeX\ primitive, or \tc{multispan}.  See notes below).  The table will be 
centered in a \tc{hbox} of width \tc{hsize}, unless you have turned table 
centering off (see the commands \tc{centeredtables} and 
\tc{noncenteredtables}).

Each time a new table is encountered, a message similar to {\tt `[Nrows=xx,
Ncols=yy]'} is printed on your terminal, where {\tt xx} is the number of rows
and {\tt yy} the number of columns discovered in your table.  If you think they
are incorrect, you may have left out some {\tt \hbox{\vbar}}'s or {\tt
\hbox{\and}}'s or \tc{cr}'s.  This diagnostic feature can be disabled by saying
\tc{tableinfofalse} (and restarted by saying \tc{tableinfotrue}).

\vbox{\vskip 0.3truein}
{\em An example}

A simple 3-row, 2-column table with a header spanning two columns could be 
specified as (see notes 3 and 4 below on the use of \tc{multispan}):
\begin{verbatim}
       \begintable
       \multispan{2}\tstrut\hfil The Top Line\hfil\crthick
       Entry 1 | Entry 2 \cr
       Entry 3 | Entry 4 \endtable
\end{verbatim}
These commands produce Table~\ref{tab:three}.
\begin{table}[h]
       \begintable
		\multispan{2}\tstrut\hfill The Top Line\hfill\crthick
		Entry 1 | Entry 2 \cr
		Entry 3 | Entry 4 \endtable
\caption{A sample table}\label{tab:three}
\end{table}
\clearpage

\vbox{\vskip 0.3truein}
{\em An example of non-centered tables}

Two or more tables can be placed side-by-side by using the
\tc{noncenteredtables} command.  Consider the two tables here
(Table~\ref{tab:four}): 

\begin{table}[h]
       \noncenteredtables
       \tableline{
       \begintable
       Item ABC | Item DEF \cr
       Item GHI | Item JKL \endtable
       \hfil
       \begintable
       Data 111 | Data 222 \cr
       Data 333 | Data 444 \cr
       Data 555 | Data 666 \endtable
       }
\caption{Two non-centered tables aligned side-by-side}\label{tab:four}
\end{table}

These were produced by saying
\begin{verbatim}
       \noncenteredtables
       \line{
       \begintable
       Item ABC | Item DEF \cr
       Item GHI | Item JKL \endtable
       \hfil
       \begintable
       Data 111 | Data 222 \cr
       Data 333 | Data 444 \cr
       Data 555 | Data 666 \endtable
       }
\end{verbatim}
Notice that tables of unequal height are aligned at the bottom.

\vbox{\vskip 0.5truein}
{\large Usage notes:}

\begin{enumerate}
\item[1.] Vertical spacing is done with a strut, called \tc{tstrut}, which 
is initially defined as 3.1ex high and 1.2ex deep.  If you don't like the 
way it looks, you can redefine \tc{tstrut} to your own liking:
\begin{verbatim}
       \def\tstrut{\vrule height hh depth dd width 0pt}
\end{verbatim}
where you specify your desired height {\tt hh} and depth {\tt dd}.
\item[2.] You can control the thickness of the thin and thick rules through
the use of \tc{thicksize} and \tc{thinsize}.  You can turn off the rules 
entirely by saying
\begin{verbatim}
       \thinsize=0pt
       \thicksize=0pt
\end{verbatim}
for example
\item[3.] If you use a \tc{multispan} or an \tc{omit} in the first column of 
a row, you will lose the effect of the \tc{tstrut} within that row and must 
specify it yourself.  See the example above.
\item[4.] Use of \tc{multispan} and \tc{omit} will cause the default 
centering of the entry to be lost; if you want it centered, put an \tc{hfil} 
on each side of the entry, as in the example above.
\item[5.] To override the default centering action, include an \tc{hfill} on 
the left or right as desired; the \tc{hfill} will override the default 
\tc{hfil}.
\end{enumerate}

\vbox{\vskip 0.3truein}
{\large Local modifications for use with \LaTeX :}

\vbox{\vskip 0.1truein}
This macro package was originally written for use with plain \TeX.  Bob Taylor
has made changes for it to be used as a style option in \LaTeX.  The plain
\TeX\ command \tc{line} has been replaced with \tc{tableline}, with 
\tc{tableline} being defined as 
\begin{verbatim}
       \def\tableline{\hbox to \hsize}
\end{verbatim}

The \tc{documentstyle} command in your \LaTeX\ file should look something like:
\begin{verbatim}
       \documentstyle[tables]{article}
\end{verbatim}

\end{document}
