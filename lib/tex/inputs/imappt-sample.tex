\magnification=\magstep1
\input amstex
\documentstyle{imappt}
%%%%%%%%%%%%%%%%%%%%%%%%%%%%%%%%%%%
% Macros specific to this paper   %
\define\BbbR{{\Bbb R}}            %
\define\loner{{L^1(\BbbR)}}       %
\define\linfr{{L^\infty(\BbbR)}}  %
\define\bvr{{BV(\BbbR)}}          %
\define\TV{{\roman {TV}}}         %
\define\sdot{\,\cdot\,}           %
%%%%%%%%%%%%%%%%%%%%%%%%%%%%%%%%%%%
\topmatter
\title
A SAMPLE PAPER TO ILLUSTRATE THE \\
IMA PREPRINT STYLE\footnote"*"{Unlikely to appear.}
\endtitle
\author
BRADLEY J. LUCIER\footnote"\dag"{Department of Mathematics, Purdue University,
West Lafayette, Indiana 47907.
The work of the first author was not supported by the
Wolf Foundation.
} and
DOUGLAS N. ARNOLD\footnote"\ddag"{Department of Mathematics, University
of Maryland, College Park, Maryland 20742.}
\endauthor
\abstract{This nonsense paper exemplifies the IMA preprint style file for \AmSTeX\,
imappt.sty.
The IMA preprint style file, which 
is used by the Institute for Mathematics
and its Applications for its preprints,
is offered as a general purpose preprint style for mathematical papers.
It is a modification of the amsppt style of
Michael Spivak.  This paper also serves to illustrate many of the amstex
macros as used with the imappt style file.}

\keywords{porous medium, interface curves}
\subjclass{65N60}
\endtopmatter
\document
\subheading{1. Introduction}
We are concerned with numerical approximations to the so-called
porous-medium equation \cite{5},
$$
\alignedat2
  &u_t=\phi(u)_{xx},&&\qquad x\in\BbbR,\quad t>0,\quad\phi(u)=u^m,\quad m>1,
\\
  &u(x,0)=u_0(x),&&\qquad x\in\BbbR.
\endalignedat
\tag 1.1
$$
We assume that the initial data $u_0(x)$ has bounded support, that
$0\leq u_0\leq M$, and that $\phi(u_0)_x\in\bvr$.
It is well known that a unique solution $u(x,t)$ of (1.1) exists,
and that $u$ satisfies
$$
  0\leq u\leq M\text{ and }\TV\phi(u(\,\cdot\,,t))_x\leq\TV\phi(u_0)_x.
\tag 1.2
$$
If the data has slightly more regularity, then this too is satisfied
by the solution. Specifically, if $m$ is no greater than two and
$u_0$ is Lipschitz continuous, then $u(\,\cdot\,,t)$ is also Lipschitz;
if $m$ is greater than two and $(u_0^{m-1})_x\in\linfr$, then
$(u(\,\cdot\,,t)^{m-1})_x\in\linfr$ 
(see [3]). (This will follow from results presented here, also.)
We also use the fact that the solution $u$ is H\"older continuous in $t$.

\subheading{2. $\linfr$ error bounds}
After a simple definition, we state a theorem
that expresses the error of approximations $u^h$ in
terms of the weak truncation error $E$.
\proclaim{Definition 2.1}\rm A {\it definition}
is the same as a theorem set in roman
type.
\endproclaim
\proclaim{Theorem 2.1}
Let $\{u^h\}$ be a family of approximate solutions satisfying
the following conditions for $0\leq t\leq T$:
\roster
\item For all $x\in\BbbR$ and positive $t$, $0\leq u^h(x,t)\leq M$;
\item Both $u$ and $u^h$ are H\"older--$\alpha$ in $x$
for some $\alpha\in(0,1\wedge 1/(m-1))$; $u^h$ is right continuous in $t$;
and $u^h$ is H\"older continuous in $t$ on
strips $\BbbR\times(t^n,t^{n+1})$, with the set $\{t^n\}$ having no
limit points; and
\item There exists a positive function $\omega(h,\epsilon)$ such that:
whenever $\{w^\epsilon\}_{0<\epsilon\leq\epsilon_0}$ is a family of functions
in $\bold X$ for which
{\roster
\item"(a)" there is a sequence of positive numbers $\epsilon$ tending
to zero, such that for these  values of
$\epsilon$, $\|w^\epsilon\|_\infty\leq 1/\epsilon$,
\item"(b)" for all positive
$\epsilon$, $\|w_x^\epsilon(\sdot,t)\|_\loner\leq 1/\epsilon^2$, and
\item"(c)" for all $\epsilon>0$, 
$$
\sup\Sb
x\in\BbbR\\0\leq t_1,t_2\leq T\endSb
\dfrac{|w^\epsilon(x,t_2)-w^\epsilon(x,t_1)|}{|t_2-t_1|^p}\leq 1/\epsilon^2,
$$
where $p$ is some number not exceeding $1$,
\endroster}
then $|E (u^h,w^\epsilon,T)|\leq\omega(h,\epsilon).$
\endroster
Then, there is a constant $C=C(m,M,T)$ such that
$$
\|u-u^h\|_{\infty,\BbbR\times[0,T]}\leq
C\left[
\sup \left |\int_\BbbR(u_0(x)-u^h(x,0))  w(x,0) \,dx\right|+
\omega(h,\epsilon)+\epsilon^\alpha\right],
\tag 2.1
$$
where the supremum is taken over all $w\in\bold X$.
\endproclaim
\demo{Proof}Let $z$ be in $\bold X$. Because $E(u,\sdot,\sdot)\equiv0$,
Equation (1.5) implies that
$$
\int_\BbbR\Delta uz|^T_0dx=\int_0^T\int_\BbbR
\Delta u(z_t+\phi[u,u^h]z_{xx})\,dx\,dt-
E(u^h,z,t),
\tag 2.2
$$
where $\Delta u=u-u^h$ and 
$$
\phi[u,u^h]=\dfrac{\phi(u)-\phi(u^h)}{u-u^h}.
$$
Extend $\phi[u,u^h](\cdot,t)=\phi[u,u^h](\cdot,0)$ for negative $t$, and
$\phi[u,u^h](\cdot,t)=\phi[u,u^h](\cdot,T)$
for $t>T$.\footnote{This is an obvious ploy, but we need a footnote.}
Fix a point $x_0$ and a number $\epsilon>0$. Let $j_\epsilon$
be a smooth function of $x$ with integral $1$ and support in 
$[-\epsilon,\epsilon]$,
and let $J_\delta$ be a smooth function of
$x$ and $t$ with integral $1$ and support in 
$[-\delta,\delta]\times[-\delta,\delta]$; $\delta$ and $\epsilon$ are
positive numbers to be specified later.
We choose $z=z^{\epsilon\delta}$ to satisfy
$$
\aligned
  &z_t+(\delta+J_\delta*\phi[u,u^h])z_{xx}=0,\qquad x\in\BbbR,\;0\leq t\leq T,
\\
  &z(x,T)=j_\epsilon(x-x_0).
\endaligned
\tag 2.3
$$
The conclusion of the theorem now follows from (2.1) and the fact that
$$
|j_\epsilon*\Delta u(x_0,t)-\Delta u(x_0,t)|\leq C\epsilon^\alpha,
$$
which follows from  Assumption 2.\qed
\enddemo
\Refs
\ref 
  \no 1
  \by K. Hollig and M. Pilant
  \paper Regularity of the free boundary for the porous medium equation
  \paperinfo MRC Tech. Rep. 2742
\endref
\ref 
  \no 2
  \by J. Jerome
  \book Approximation of Nonlinear Evolution Systems 
  \publ Academic Press 
  \publaddr New York 
  \yr 1983
\endref
\ref
  \no 3
  \manyby R. J. LeVeque
  \paper Convergence of a large time step generalization of Godunov's method 
         for conservation laws
  \jour Comm. Pure Appl. Math.
  \vol 37 
  \yr 1984
  \pages 463--478
\endref
\ref
  \no 4
  \bysame
  \paper A large time step generalization of Godunov's method for systems
         of conservation laws
  \jour
  \toappear
\endref
\ref 
  \no 5
  \by B. J. Lucier
  \paper On nonlocal monotone difference methods for scalar conservation laws
  \jour Math. Comp.
  \vol 47
  \yr 1986
  \pages 19--36
\endref
\enddocument
