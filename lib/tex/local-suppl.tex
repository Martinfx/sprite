\subsubsection{The {\tt a4} and {\tt a4wide} Style Options}

By default, the output page is designed to be printed on US paper.
The {\tt a4} and {\tt a4wide} style options result in output
that fits better onto A4 size pages.
It is sometimes convenient to be able to fit more text onto an A4 page.
The {\tt a4wide} option accomplishes this by using narrower margins.

\subsubsection{The {\tt tgrind} Style Option}{\label{tagrind}}

It is useful to be able to include program fragments as figures in a
text.  The {\tt tgrind} style option can be used to accomplish this
fairly painlessly by defining the \verb|\tagrind| command. 

The \verb|\tagrind| command has the following proforma:
\begin{itemize} \tt
\item[]
\verb|\tagrind*|[{\it loc\/}]\{{\it file\/}\}\{{\it caption \/}\}\{{\it label \/}\}
\end{itemize}

As usual,
\begin{itemize}
\item in two-column format, the ordinary form produces a
single-column listing and the $\ast$-forms produces a double-column listing;

\item {\it loc\/}
is a sequence of one to four of the letters \verb|h| ({\it here\/}),
\verb|t| ({\it top\/}), \verb|b| ({\it bottom\/}) and \verb|p| ({\it
page\/}) specifying where the listing may be placed. 

\item {\it file} specifies the file from which the ground listing
is to be taken.

\item {\it caption} produces a numbered caption.

\item {\it label} assigns a label to the listing, so that it may be
referred to by means to the \verb|\ref{|{\it label\/}\verb|}| command.

\end{itemize}

The listing file should be generated using {\tt tgrind} and removing the first and last lines of the resulting file. For example, if the C source file {\tt file.c} were to be used as a listing,
\begin{verbatim}
     tgrind -f -lc file.c | sed -e 1d -e \$d >file.tex
\end{verbatim}
would generate a file {\tt file.tex} suitable for use
with \verb|tagrind|. For example,
\begin{verbatim}
\tagrind{file}{This is an example}{eg}
\end{verbatim}

\subsubsection{The {\tt sfwmac} Style Option}

The {\tt sfwmac} style option defines a set of macros which are
useful when writing program documentation.  The \verb|\pgm{|{\it
program\/}\verb|}| command expects a program name as an argument.
The name in printed (in italics) and an index entry is made for it.
Similarly, the \verb|\man{|{\it program\/}\verb|}(|{\it
section\/}\verb|)| command may be used to produce references to
manual page entries.
The \verb|\arg{|{\it argument\/}\verb|}|, \verb|\switch{|{\it
switch\/}\verb|}| and \verb|\file{|{\it file\/}\verb|}| commands may be
used when referring to program arguments and switches  and to
filenames. These produce no index entry.

This style also defines a number of common program names such as
\latex/, \slitex/, etc.  These are produced by means of
\verb|\LaTeX|, \verb|\SLiTeX| commands and indexed.  

Suggestions for additions to the list of defined program names are
welcome.

\subsubsection{The {\tt trademark} Style Option}

The {\tt trademark} style option defines a number of trademarks and
trade names.  For example, \Unix/ (note the automatically defined
footnote) was produced by a \verb|\Unix/| command.  Note that there
is no need to append \verb*|\ | or \verb|{}| to prevent the
swallowing up of succeeding spaces.

Suggestions for additions to the list of trademarks are welcome.

\subsubsection{The {\tt lcustom} Style Option}

This style contains a whole bunch of useful little command
definitions.  The \verb|\inputverbatim{|{\it file\/}\verb|}|
command inserts the file in verbatim mode.  However, it is usually
better to use the \verb|\tagrind| command described in section
\ref{tagrind} for this purpose.  A number of further commands
provide a clean way of introducing diagrams and tables from
subsidiary files.  They are similar in form and usage to the to the
\verb|\tagrind| command.

\subsubsection{The {\tt vdm} Style Option}

The {\tt vdm} style option aids typesetting of VDM specifications: it
provides macros for typesetting formulae, data types, functions,
operations and proofs in whatever the currently approved VDM manner
happens to be.  A copy of the user manual for the {\tt vdm} style option
can be obtained by typing
\begin{verbatim}
     latex /usr/lib/tex/macros/vdm.tex
\end{verbatim}
