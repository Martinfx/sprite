\documentstyle{siam}
\title{Using the SIAM \LaTeX\ Style%
	\thanks{This document was written September 24, 1986.}}
\author{Howard Trickey%
	\thanks{AT\&T Bell Laboratories, 600 Mountain Ave.,
		Murray Hill, NJ 07974.
		{\tt research!trickey} or {\tt trickey@btl.csnet}}}
\begin{document}
\maketitle
\begin{abstract}
This note explains how to use a \LaTeX\ style that produces a good
approximation to the style used in SIAM journals.
The note is itself an example of the {\tt siam} style in action.
\end{abstract}
\begin{keywords}
Document preparation, \LaTeX, SIAM
\end{keywords}

\section{Introduction}
The {\em Society for Industrial and Applied Mathematics (SIAM)\/}
publishes a number of journals, as shown in Table~\ref{pub-table}.
Authors who have prepared their articles using \LaTeX\ \cite{latexbook}
can get them formatted in a style identical to a typical paper
in one of those journals.
Input that is suitable for the \LaTeX\ {\tt article} style needs
only a couple of modifications besides changing the document style
to {\tt siam}.

\begin{table}\label{pub-table}
\caption{SIAM Publications}
\begin{center}
\begin{tabular}{l}
\hline
SIAM Journal on Algebraic and Discrete Methods\\
SIAM Journal on Applied Mathematics\\
SIAM Journal on Computing\\
SIAM Journal on Control and Optimization\\
SIAM Journal on Mathematical Analysis\\
SIAM Journal on Numerical Analysis\\
SIAM Journal on Scientific and Statistical Computing\\
\hline
\end{tabular}
\end{center}
\end{table}

This paper itself gives an example of the {\tt siam} style.
Figure~\ref{outline-fig} is an outline of the input used.

\begin{figure}
\begin{center}
\begin{verbatim}
\documentstyle{siam}
\title{Using the SIAM \LaTeX\ Style%
        \thanks{This document was written September 24, 1986.}}
\author{Howard Trickey%
        \thanks{AT\&T Bell Laboratories, 600 Mountain Ave.,
                Murray Hill, NJ 07974.
                {\tt research!trickey} or {\tt trickey@btl.csnet}}}
\begin{document}
\maketitle
\begin{abstract}
This note explains ...
\end{abstract}
\begin{keywords}
Document preparation, \LaTeX, SIAM
\end{keywords}

\section{Introduction}
...
\section*{Acknowledgment}
\bibliography{siam}
\bibliographystyle{siam}
\end{document}
\end{verbatim}
\end{center}
\caption{Input used to produce this paper}\label{outline-fig}
\end{figure}

\section{Title Material}
Use the \LaTeX\ \verb|\title| and \verb|\author| declarations and the
\verb|\maketitle| command as usual.
SIAM journals don't put author affiliations below or beside
the author(s); instead, use \verb|\thanks| to list addresses.
In fact, it is an error to use \verb|\\| in the \verb|\author|
command, because it interferes with some fairly fancy
footwork that the style uses to change the author and title
to all uppercase.
If there are several authors with the same affiliation,
the best I can suggest is to \verb|\thanks| only one, see
what footnote mark comes out, and manually place it after
the others (sorry).

The \verb|\maketitle| can be followed by an abstract and/or a keywords
section, as shown in the introduction.

\section{Figures and Tables}
SIAM tables have their captions above the table contents,
while figures have captions below the contents.
Place the \verb|\caption| command accordingly.
Figure~\ref{outline-fig} and Table~\ref{pub-table}
were examples.

\section{Mathematics}
Equations are entered as usual.
SIAM journals put equation numbers on the left.
It appears to be up to the author whether equations are numbered
within a section (e.g., (4.2)) or within the paper (e.g., (2)).
The {\tt siam} style uses the latter, as this example from
the {\em SIAM Journal on Numerical Analysis\/}
\cite{bube} shows:
\begin{equation}
r_k=\int_{-\Delta}^{\Delta} r(2k\Delta+\tau)\,d\tau\qquad (0\le k \le n)
\end{equation}

There are four ``theorem-like'' environments predefined:
theorem, lemma, corollary, and proposition.
Again, SIAM does not appear to prescribe the exact numbering scheme
to be used.
The {\tt siam} style numbers them all in the same index space, and
not within each section.
There is also a \verb|\proof| environment.
For example, here is Theorem~1 of the example paper \cite{bube}:
\begin{theorem}
Suppose the ramp reflectance is sampled. Then
\begin{equation}\label{theq}
\|R_j\|_2 \le c_0 \equiv \|R_X\| < 1
\end{equation}
whenever $(j+1)\Delta\le X$;
thus the discrete inverse problem can be solved,
yielding a positive discrete impedance profile.
\end{theorem}
\begin{proof}
Fix $\Delta$, and fix $j$ and $x$ with $0<x=(j+1)\Delta \le X$.\\
\dots so since $R_j$ is symmetric, (\ref{theq}) follows.
\end{proof}

\section{SIAM Bibliographies}
The {\tt siam} bibliography style uses numeric labels, abbreviations
from Mathematical Reviews, titles in italics, mixed upper-lower case
for book titles, lower case for article titles, and uses commas
to separate fields.
SIAM journal articles sometimes order the entries alphabetically,
sometimes order-of-reference.
The {\tt siam} style does alphabetic ordering.

\section{Page Styles}
Use the {\tt plain} style, or perhaps {\tt empty} or {\tt myheadings}.
The {\tt headings} style doesn't do the right thing.
(A problem with making {\tt headings} work is that
often the title and/or authors have to be abbreviated to fit
in the headings.)

\section*{Acknowledgment}
Eric Grosse and Bill Coughran were the instigators and first users
of this style.
Eric has made it available via {\em netlib}, a repository
of mathematical software.  You can get the {\tt siam} style
files by
\begin{verbatim}
    mail research!netlib
    send latex from siam
\end{verbatim}

\bibliography{siam}
\bibliographystyle{siam}
\end{document}
