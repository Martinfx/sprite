% Version of 23 July 1985
\documentstyle[titlepage]{article}

%&t&{\tt #}&
%&h&\hbox#& 
%&v&\hbox{\verb"#"}&
%&m&\mbox{#}& 
%&?&-@s:\subsubsection*: 0l .u1 2l @s:\subsubsection*: 0l q1,.xa q1,.k#&

\nofiles
\def\LATEX{\LaTeX}
\let\TEX = \TeX
\def\BIBTEX{BIB\kern-.1em\TeX}
\def\SLITEX{SLI\TeX}

\begin{document}

\title{\LaTeX\ Update}

\date{Version 2.09 --- \today
\\[40pt]
\begin{minipage}{.8\textwidth}
\normalsize
\noindent
This corrects all known significant inaccuracies
in the December 13, 1983 Version of the \LaTeX\ manual, including
both typographic errors and changes to \LaTeX.\\[10pt]
This document was extensively revised on November~8, 1984.  After this
revision, all subsequent material will be added only at the end.\\[10pt]
References to pages in the manual refer to the December~13, 1984
printing.  Subsequent printings have slightly different pagination.
Line $-n$ denotes the $n^{th}$ line from the bottom of the page.
\end{minipage}}

\author{Leslie Lamport}

\maketitle




\section*{Errata}

There are many typos and omissions in the preliminary manual.  Most of them
are obvious and you can correct them yourself.  The following
are the only ones that may cause trouble.


\subsubsection*{page 10, line -9}
Typing \hbox{\verb|<|} and \hbox{\verb|>|} do not produce the
expected results.  These characters should be typed only in
math mode.


\subsubsection*
{page 22, line 9}
``right to left'' should read ``left to right''.

\subsubsection*
{page 41}
The manual should mention that
a list environment must contain at least one \verb"\item"
command.

\subsubsection*
{page 48, second paragraph}
Ignore this paragraph.  There are cases in which omitting the braces
can cause trouble.


\subsubsection*
{page 77, lines 18 and 22}
Delelete the \hbox{\verb|#|}; the optional argument should just be
{\tt 1}.

\subsubsection*{page 79, line -6}
Replace \verb"\textwidth" by \verb"\textheight".

\subsubsection* {page 90}
If a \hbox{\verb"minipage"} environment contains nothing but one or more
\hbox{\verb"tabbing"} environments, then the width of the resulting
box is the minimum of: (i)~the argument of the \hbox{\verb"minipage"}
environment and (ii)~the width of the widest line in the
\hbox{\verb"tabbing"} environment(s).  This may be useful for
making a \hbox{\verb"\parbox"} whose width is determined by
its contents.

\subsubsection*{page 103ff}
An \hbox{\verb"array"} and \hbox{\verb"tabular"} environment leaves
a small space (half the normal intercolumn space) at the left
and right of the array.  A \hbox{\verb"|"} or {\tt @}-expression
at the beginning (end) of the argument suppresses the space at
the left (right).  This space usually does no harm, but can be
eliminated by putting \hbox{\verb"@{}"} at the beginning and end
of the argument.

\subsubsection*{page 118, line -12}
There should be three more spaces before ``{\tt Special characters:}''.

\subsubsection*{Page 128}
Add a reminder about typing \hbox{\verb|\twocolumn[{ ... }]|} if the
argument contains a {\tt ]}.

\subsubsection*{Page 138}
Replace \hbox{\verb|\bibstyle|} by \hbox{\verb|\bibliographystyle|}.

\section*{Changes Through Version 2.05}

\subsubsection*{pages 46ff}
A new \mbox{\tt eqnarray} environment has been added.
Writing
\begin{verbatim}
       \begin{eqnarray} ... \end{eqnarray}
\end{verbatim}
is pretty much the same as writing
\begin{verbatim}
       \[ \begin{array}{lcl} ... \end{array} \]
\end{verbatim}
except that:
\begin{itemize}
   \item An equation number is put on every line, unless there
         is a \hbox{\verb"\nonumber"} command on that line.
         
   \item Page breaking can occur between lines, so the {\tt *}-form
         of the \verb"\\" command is useful.
   \item There is an extra bit of space added between the lines.
         The amount of space is equal to the length \verb"\jot",
         which is a parameter you can set.
\end{itemize}
There's also a \hbox{\tt eqnarray*} environment which is the
same as \hbox{\tt eqnarray} except it doesn't put in any equation
numbers.  

The {\tt equation}, {\tt eqnarray} and \hbox{\tt eqnarray*} environments
are called {\it equation\/} environments.  You can put them inside
parboxes, but you should not put an equation environment inside a 
parbox that's inside an equation environment.

\subsubsection* {Section 2.3.2}
The following math symbols have been added.  (Like all math symbols,
they can be used only in math mode.)
\[ \begin{tabular}{ll@{\protect\hspace{20pt}}ll}
    $\mho$ & \verb|\mho| & $\sqsubset$ & \verb|\sqsubset|\\
    $\Join$ & \verb|\Join| & $\sqsupset$ & \verb|\sqsupset|
\end{tabular}
 \]

\subsubsection*{page 48}
Upper-case Greek letters come in the same type styles as ordinary
Roman letters.  In math mode you can type \hbox{\verb"{\bf \Pi}"}
to produce ${\bf \Pi}$, \hbox{\verb"{\tt \Pi}"} to produce
${\tt \Pi}$, and so on.  

\subsubsection*{pages 55, 72ff, and 142}
You can now put ``dangerous'' commands in an {\tt @}-expression, or
in the argument of a sectioning or \hbox{\verb|\caption|} command, by
preceding the command with \hbox{\verb|\protect|}.  For example, you
can write 
\begin{verbatim}
       \begin{array}{l@{\protect\makebox[.2in]{=}}l} ...
\end{verbatim}
or
\begin{verbatim}
       \section{Figure \protect\ref{foo} Re-examined.}
\end{verbatim}
The \hbox{\verb|\writecommand|} command has been eliminated, and
\hbox{\verb|\protect\foo|} does what \hbox{\verb|\writecommand{\foo}|}
used to.

The \hbox{\verb"\protect"} command can be used in the argument of a
\hbox{\verb"\typeout"} or \hbox{\verb"\typein"} command, so
\begin{verbatim}
       \typeout{Command \protect\foo?}
\end{verbatim}
causes \TEX{} to type
\begin{verbatim}
       Command \foo?
\end{verbatim}
on the terminal.  Up until version~2.07, you can also use
\hbox{\verb"\protect"} in the argument of an \hbox{\verb"\index"}
command to write a command onto the {\tt .IDX} file.  
(See the change to \hbox{\verb"\index"} in version~2.07.)

An extra \hbox{\verb"\protect"} command does no harm, so you can use
one when you're not sure if it's necessary.

\subsubsection*{page 60}
The declaration \hbox{\verb"\boldmath"} has been added.  It changes
the math italic typeface and the math symbols to boldface.  (This
makes lowercase Greek letters bold, but not uppercase ones.)  The
declaration \hbox{\verb"\unboldmath"} undoes its effect.  These
declarations are like size-changing declarations, and they should {\it
not\/} be used in math mode.  The \hbox{\verb"\boldface"} declaration
has some anomalies in terms of what it makes bold and what it
doesn't.  Also, it does not embolden subscripts and superscripts.

\subsubsection*{page 99}
A \hbox{\tt tabbing} environment should not appear inside
another \hbox{\tt tabbing} environment.  (This would
be possible only if the inner environment were inside a
parbox.)

\subsubsection*{pages 117-118}
{\it This change has been made obsolete in version~2.07 by the
\hbox{\verb"\newenvironment"} command.}\\
The \hbox{\verb"\newlist"} command now has an optional argument that
may come right before the last argument.  It specifies the number of
arguments that the new list environment has.  (The default is, as
before, no arguments.)  These arguments can be mentioned in the last
mandatory argument just as in the 
\hbox{\verb"\newcommand"} command.  For example,
you can write:
\begin{verbatim}
       \newlist{labelquote}{}{}[1]{\item {\bf #1:} }
\end{verbatim}
to define a \hbox{\tt quote}-like environment which typesets its
argument in boldface in front of the \hbox{\tt quote}'d material.

\subsubsection* {page 125} 
In \LATEX{} version 2.01, the \hbox{\verb"\mainbaselineskip"} command
has been eliminated.  In its place is the command
\hbox{\verb"\baselinestretch"} that works as follows.  Every command
that changes the type size sets the value of
\hbox{\verb"\baselinskip"} to \hbox{\verb"\baselinestretch"} times its
``ordinary'' value for that type size.  For example, the ordinary
\hbox{\verb"\baselineskip"} for a ten-point font is 12pt.  If
\hbox{\verb"\baselinestretch"} equals 1.5 and the
\hbox{\verb"\normalsize"} type size is ten-point, then the
\hbox{\verb"\normalsize"} command sets \hbox{\verb"\baselineskip"} to
$1.5\times 12$pt, or 18pt.

You can change \hbox{\verb"\baselinestretch"} at any time with a
\hbox{\verb"\renewcommand"} declaration, but remember that this has no
immediate effect on the value of \hbox{\verb"\baselineskip"}; it 
affects only later type-size-changing commands.  The 
\linebreak
\hbox{\verb"\begin{document}"} command executes a
\hbox{\verb"\normalsize"} command to initially set the type size, so
changing \hbox{\verb"\baselinestretch"} before the
\hbox{\verb"\begin{document}"} command has the effect of
``magnifying'' \hbox{\verb"\baselineskip"} throughout the document.


\subsubsection*{page 138}
The \hbox{\tt thebibliography} environment has been changed slightly.
The \verb"\bibitem" command automatically puts brackets or whatever
around the labels in a bibliography, so the example on line -7 should
read
\begin{verbatim}
       \bibitem[Knuth83]{knuth-tex82} Knuth, Donald ...
\end{verbatim}
The document style determines if the label is to be printed as
``[Knuth 83]'' or ``Knuth 83.'' or whatever.  The argument of
the environment itself should be the same as
the optional argument of a \hbox{\verb"\bibitem"} command to generate the
longest label.


\subsubsection*{page 146}
The error message 
\begin{verbatim}
       ! Something's wrong--perhaps a missing \item.
\end{verbatim}
has been added.  The most likely cause is forgetting an
\hbox{\verb|\item|} command in a {\tt list} environment.  It is also
caused by forgetting the argument of a \hbox{\tt thebibliography}
environment.

\subsubsection*{page 146}
A new \TEX{} error is now possible:
\[ \mbox{\tt ! \ Font ... not loaded: Not enough room left.} \]
This means that you are using too many different typefaces, where
using the \hbox{\verb"\it"} type style in \hbox{\verb"\small"},
\hbox{\verb"\normalsize"} and \hbox{\verb"\large"} counts as three
different typefaces.  You should probably eliminate some of those
typefaces, since they're just going to confuse the reader anyway.
However, you might solve your problem by \LATEX{}ing your
document in parts.


\subsubsection*{page 148}
The error message
\begin{verbatim}
     !  This is a LaTeX bug.
\end{verbatim}
has been changed to
\begin{verbatim}
     !  This may be a LaTeX bug.
\end{verbatim}
It is possible, though difficult, to get it with bad input.

\subsubsection*
{pages 148,152}
The ``{\tt Typeface not available}'' error has been eliminated.  If a
typeface isn't available, \LATEX{} uses a different font and types out
the warning:
\[  \mbox{\tt No ... typeface in this size, using ...}\] 

\subsubsection*{Page 160 (addendum)}
Text in which there is no color declaration in effect appears on all
color layers.  For example, if you make no color declarations anywhere
in your slide file, then all color layers will be identical to the
black and white versions.  Note that color declarations are undefined
in the root file.

\subsubsection*{Page 161 (addendum)}
The warning about horizontal lines appearing in color layers where
they should not can be eliminated.  Colors should now always be
handled properly.

\section*{Changes Made in Version 2.06}

\subsubsection*
{pages 8, 119, 120, etc.:}
The \hbox{\verb|\pagelayout|} command has been eliminated; the parameters
that used to be specified by the page layout are now specified by
the documentstyle.  The \hbox{\verb|\documentstyle|} command has been
changed.  It now has the following form:
\begin{quote}
\tt \verb|\documentstyle|[{\it 
   options\/}]\hbox{\verb"{"}{\it style\/}\hbox{\verb"}"}
\end{quote}
where {\it style\/} is the main style and {\it options\/} is an
optional list of options, separated by commas.  The main styles 
and available options are:
\[ \begin{tabular}{ll}
    \multicolumn{1}{c}{\underline{{\it Main Style\/}}} & 
      \multicolumn{1}{c}{\underline{{\it Options\/}}}\\
    \tt report & \tt 11pt, 12pt, twoside, twocolumn, draft, fleqn\\
    \tt article & \tt 11pt, 12pt, twoside, twocolumn, draft, fleqn\\
    \tt letter &  \tt 11pt, 12pt, fleqn\\
    \tt slides\footnotemark &   \multicolumn{1}{c}{---}
   \end{tabular}\]
\footnotetext{\SLITEX\ only.}%
The \hbox{\verb"fleqn"} option makes all displayed equations
flushleft, a distance of \linebreak%%%%%
\hbox{\verb"\mathindent"} from the prevailing
left margin.

\subsubsection*{page 10}
You can forget about the \hbox{\verb"\mbox"} nonsense for producing
a sentence-ending period.  You simply type \hbox{\verb|"|} (that's
a double-quote) {\it before\/} the period, so the example should
be typed as follows:
\begin{verbatim}
     Euclid et al.\ proved I + I = II". Meanwhile ...
\end{verbatim}
You can use the \hbox{\verb|"|} in this way with other punctuation
marks that ordinarily have extra space added after them, including
\hbox{\verb"?"} and \hbox{\verb":"}.  You should also do the same
thing before a closing parenthesis, as in:
\begin{verbatim}
     ... I + I = II"?)  Meanwhile ...
\end{verbatim}
Note that you should not have any other \hbox{\verb|"|}s in your
text (except inside a \hbox{\verb"verbatim"} environment or
a \hbox{\verb"\verb"} command).\\
{\it In version 2.07, you type \verb|\@| instead of \verb|"|.}

\subsubsection*{pages 11, 72ff}
Two new levels of sectioning below \hbox{\verb"\subsubsection"} have
been added: they are \hbox{\verb"\paragraph"} and \hbox{\verb"\subparagraph"},
together with their $*$-forms.  Two new counters have also been added
to control which levels of sectioning are to be numbered and which are
to be listed in the table of contents.  Setting the
counter \hbox{\verb"secnumdepth"} to $k$ ($k\ge 1$) means that section
levels down through level $k$ (inclusive) are numbered, and setting
the counter \hbox{\verb"tocdepth"} to $k$ means that sections at
levels down through $k$ are to appear in the table of contents.  In
both the \hbox{\verb"report"} and \hbox{\verb"article"} document
style, \hbox{\verb"\section"} is the level-1 command,
\hbox{\verb"\subsection"} the level-2 command, and so on.  (The
restrictions on what may appear in the argument of a sectioning
command given on page~72, as modified by the introduction of the
\hbox{\verb"\protect"} command, apply even if \hbox{\verb"tocdepth"}
specifies that no table-of-contents entry is to be produced.)

\subsubsection*{pages 19, 32}
The declaration \hbox{\verb"\em"} 
chooses a font style for \mbox{\tt em\it phasis}---either
italic or roman, depending upon where it is used.  Thus, typing
\begin{verbatim}
     Remember: {\em this is {\em very} important.}  
\end{verbatim}
produces
\begin{quote}
Remember: {\em this is {\em very} important.}
\end{quote}
It is best to use \hbox{\verb"\em"} intead of \hbox{\verb"\it"}
for two reasons: (i)~you might later decide to italicize
a paragraph and forget that to convert \hbox{\verb"\it"} commands
to \hbox{\verb"\rm"} commands, and (ii)~there are environments
like the \hbox{\tt theorem} environment which some document styles
might set in roman type and others in italic.

\subsubsection* {pages 41,42} 
The \hbox{\tt description} environment no longer takes an argument.
It also formats things differently---the label is now italicized and
flushed right.  You aren't supposed to worry about the
formatting---\LATEX\ is supposed to take care of that for you.
However, if you must fiddle with it, look up the definition
of the \hbox{\verb"description"} environment in the {\tt .DOC}
versions of the style files.

\subsubsection*{pages 54, 55, 72, 84}
The \hbox{\verb"\s"} command has been eliminated.  You can now use
an \hbox{\verb"\hspace"} command wherever you had to use the \hbox{\verb"\s"}
command.

\subsubsection* {page 75}
A new feature for making a title has been added.  The command
\hbox{\verb"\maketitle"}, which makes a title for your document which
goes on a separate page for the \hbox{\verb"report"} style and goes at
the top of a page for the \hbox{\verb"article"} style.  (The
\hbox{\verb"titlepage"} option for the \hbox{\verb"article"} style
causes the title to be on a separate page.)  The components of the
title are declared with the following commands:
\begin{itemize}
  \item \verb"\title" declares the title.  You may use
    \hbox{\verb"\\"} to generate a new line.
  \item \verb"\author" declares the author.  Use \hbox{\verb"\\"}
    to start new lines---for example, between the author's name and his
    address.
  \item \verb"\date" declares the date.  If omitted, today's date
        is used.
\end{itemize}
A \hbox{\verb"\thanks"} command may be used in the title, author, or
date.  It acts pretty much like a \hbox{\verb"\footnote"} command,
and is usually used to attach an acknowledgement to an author's name.
These declarations may come anywhere before the \hbox{\verb"\maketitle"}
command, as in the following example.
\begin{verbatim}
     \title{This is Your Life}
     \author{John Jones\thanks{Work supported by ...} and Fred
             Smith\thanks{Work supported by ...} \\ 
             Institute for Inhuman Development}
     ...
     \begin{document}
     \maketitle
\end{verbatim}
You can use at most one \hbox{\verb"\maketitle"} command in your
document.\\
{\it See also the \hbox{\verb"\and"} command for\/ multiple
authors and the \hbox{\verb"abstract"} environment, defined in Version~2.07.}

\subsubsection*{page 82}
The \hbox{\verb"\newnumbering"} command has been eliminated.  I will
illustrate how counter numbering works using the \hbox{\verb"section"}
counter as an example.  The section number, as it appears in the
table of contents or in the section heading, is produced by the
command \hbox{\verb"\thesection"}.  You can change how sections
are numbered by redefining this command.  For example, assuming
that chapters are numbered I, II, $\ldots$, to get sections
numbered II.A, II.B, II.C, $\ldots$ you can type the following
\begin{verbatim}
    \renewcommand{\thesection}{\thechapter.\Alph{section}}
\end{verbatim}

\subsubsection*{page 88}
The naming of ``bins'' for saving boxes with the
\hbox{\verb"\savebox"} command has been changed to conform to the same
convention as the naming of lengths.  There is a
\hbox{\verb"\newsavebox"} declaration that defines a bin.  Instead of
writing a bin number, the \hbox{\verb"\savebox"} and
\hbox{\verb"\usebox"} commands take a bin name.  For example, you
write
\begin{verbatim}
     \newsavebox{\foo} ...
     \savebox{\foo}[1in]{gnus} ...
     \usebox{\foo}
\end{verbatim}
When you leave the scope of a \hbox{\verb"\newsavebox"} declaration,
the contents of the box vanishes.  This is a handy way to reclaim the
space taken up by the contents of a bin.  However, if you don't put
the \hbox{\verb"\newsavebox"} command  ``downstairs'', but define
it at ``ground level'', then remember to 
``empty'' the bin when you don't need its contents any more
by a command like
\begin{verbatim}
     \savebox{\foo}{}
\end{verbatim}
The old style \hbox{\verb"\savebox"} and \hbox{\verb"\usebox"}, with
bin numbers, will still work as before, so you don't need to change
your current files.  But you should use the new conventions in the
future.

\subsubsection*{page 90}
Eliminate the second warning---the one about beginning or ending a
\mbox{\tt minipage} environment with a paragraph-making environment.
You can now put other environments inside a \mbox{\tt minipage} with
no problems.

\subsubsection*{page 115ff}
The \hbox{\verb"list"} environment has been changed in the
following ways.
\begin{enumerate}
\item The \hbox{\verb"\topsepcorrection"} parameter has been eliminated
and a new (rubber) length \hbox{\verb"\partopsep"} added.
The space inserted above and below the environment
is determined as follows:
\begin{itemize}
 \item If it begins a new paragraph, then 
       $\hbox{\verb"\topsep"}+\hbox{\verb"\partopsep"}$.
  \item If not, then \hbox{\verb"\topsep"}.
\end{itemize}

\item The default settings of all the list parameters are now
   determined by the document style, and may vary with the nesting level
   of the list---e.g., being different for a list inside a list than for
   a top-level list.  By convention, \hbox{\verb"\leftmargini"},
   $\ldots$~, \hbox{\verb"\leftmarginvi"} still determine the default
   value for \hbox{\verb"\leftmargin"}.
\end{enumerate}


\subsubsection*{page 120}
There are now two versions of every document-style file.  The \TeX\
``code'' for the \hbox{\verb"report"} document style still resides on
the file \hbox{\verb"report.sty"}, but there is also a file
\hbox{\verb"report.doc"} that has essentially the same code completely
documented.  (The file \hbox{\verb"report.sty"} is obtained from
\hbox{\verb"report.doc"} by eliminating comments and extra spaces, and
by adding some optimizations that make the \TeX\ code less readable.)
The \hbox{\verb".doc"} files will make it easier for you to figure out
how to customize your own document style.

\subsubsection*{page 120}
A new command \hbox{\verb"\raggedbottom"} has been added.  It causes
pages to assume their natural height instead of stretching them to all
be the same height.  It can be used anywhere, but it makes little
sense to put it anywhere except before the
\hbox{\verb"\begin{document}"} command.\\
{\it See the further amplification of this feature in Version~2.07.}

\subsubsection*{page 151}
Add the following error messages:
\begin{verbatim}
   ! \textfont ... is undefined (character ...).
   ! \scriptfont ... is undefined (character ...).
   ! \scriptscriptfont ... is undefined (character ...).
\end{verbatim}
This is actually a \LaTeX\ bug which has not been corrected because it
happens very infrequently, and eliminating it would require adding some
unpleasant restrictions to \LaTeX.  It occurs when using certain
uncommon type-style declarations, like \hbox{\verb"\sc"}, in math
mode.\\
{[}{\it The following has been made obsolete by the
\hbox{\verb|\load|} command introduced in Version~2.08.}]\\
To get around the problem, you need simply add a ground-level
(not inside any environment) declaration of the offending type style
before the formula---preferably at ground level.  Thus, you can
put the commands \hbox{\verb"\sc \rm"} somewhere before a math environment
that uses the \hbox{\verb"\sc"} command.   The ``dummy'' \hbox{\verb"\sc"}
declaration should occur at the same type size as the one in the
math environment, so a problem caused by
\begin{verbatim}
       {\large $ ... {\sc foo} ... $ }
\end{verbatim}
can be corrected by adding \hbox{\verb"\large \sc \normalsize"}.

\section*{Changes in Version 2.06a}

\subsubsection* {nowhere in particular}
The command \hbox{\verb"\TeX"} produces the \TeX\  logo, and
\hbox{\verb"\LaTeX"} produces the \LaTeX\ logo.


\section*{Changes in Version 2.07}

\subsubsection*{page 10}
As noted above, the command \verb|"| introduced in version 2.06 has
been renamed \verb|\@|.  Thus, you type
\begin{verbatim}
       Euclid et al.\ proved I + I = II\@. Meanwhile ...
\end{verbatim}
to get the proper end-of-sentence space after the period.

\subsubsection*{page 36}
The command \hbox{\verb"\pounds"} produces a \pounds\ (pounds sterling)
symbol.


\subsubsection*{page 52, Section 2.3.3 and pages 9-10}
The \hbox{\verb|\,|} command will work outside math mode, where it also
produces a ``thin'' space.  It is useful for quotes inside quotes,
typing \hbox{\verb|``\,`foo'\,''|} to get ``\thinspace`foo'\thinspace''.

\subsubsection*{page 59}
The \hbox{\verb"\ldots"} command now works in any mode, not just in
math mode.

\subsubsection* {page 62ff}
You can now cross-reference footnotes by putting a \hbox{\verb"\label"}
command in with the text of the footnote, as in
\begin{verbatim}
this is it.\footnote{\label{its-it}It really is.}
\end{verbatim}

\subsubsection* {page 75}
A new feature has been added to the title-making macros.  Multiple
authors in an \hbox{\verb"\author"} command should be separated by
an \hbox{\verb"\and"} command, as in
\begin{verbatim}
\author{T. Jones \\ SRI International \and J. Smith \\ Stanford}
\end{verbatim}
Note that multiple-line ``authors'' are made with the \hbox{\verb"\\"}
command, as expected.

There is also an \hbox{\verb"abstract"} environment for producing the
abstract of the paper.  It should normally follow the
\hbox{\verb"\maketitle"} command.  You can still roll your own
abstract if you want, but the \hbox{\verb"abstract"} environment makes
life easier when changing document styles.

\subsubsection*{pages 81ff, 115ff}
The method of implementing label numbers in lists has changed, making
it easier to define your own list environments.  The {\tt list}
and \hbox{\verb"enumerate"} counters have been eliminated, and two new
commands have been added.  First, there's a \hbox{\verb"\newcounter"}
command, where \hbox{\verb"\newcounter{foo}"} defines a new counter
named {\tt foo}.  There's an optional second argument, where writing
\begin{verbatim}
       \newcounter{foo}[subsection]
\end{verbatim}
causes the counter {\tt foo} to be reset to zero whenever the
{\tt subsection} counter is incremented.

The second relevant command has the form \hbox{\verb"\usecounter{foo}"},
where {\tt foo} is any previously defined counter.  This command is used
in the second argument of the {\tt list} environment to allow counter
{\tt foo} to be used to number the list items.  More precisely, it
causes {\tt foo} to be initialized to zero and incremented by every
\hbox{\verb"\item"} command that has no optional argument.  A 
\hbox{\verb"\label"} command within an item will refer to the value
of counter {\tt foo} for that item.

The {\tt enumerate} environment is implemented with the counters {\tt
enumi} for an outermost environment, {\tt enumii} for the first nested
inner {\tt enumerate}, $\ldots$ and {\tt enumiv} for a fourth-level
enumeration.

If you've played around with the document styles to modify any of
the list environments, you'll have to look at the new {\tt .doc} files
to see how the inner workings have been changed.  (It's now easier to
modify the way enumerations are numbered and referenced.)

\subsubsection* {page 82}
In addition to \hbox{\verb"\arabic"} \ldots\ \hbox{\verb"\Alph"},
there's a new \hbox{\verb"\fnsymbol"} command for printing counter
values.  It produces the the following sequence of nine symbols:
\[ 
* \ \ \dagger \ \ \ddagger \ \
   \mathchar "278 \ \ \mathchar "27B \ \ \| \ \ ** \ \ \dagger\dagger
   \ \ \ddagger\ddagger
 \]
that can be used for numbering footnotes.  (The double symbols look
better on footnotes than they do here.)  Warning: the symbols it
produces can be used only in math mode, so if you want to
cross-reference something that's numbered in this way, you have to put
the \hbox{\verb"\ref"} command in math mode.


\subsubsection*{page 102, line -5}
You will be able to type \hbox{\verb|\a`|}, \hbox{\verb|\a'|},
and \hbox{\verb|\a=|} to produce these accents inside a tabbing
environment.

\subsubsection*{page 104} 
A new command has been added for drawing horizontal lines across
only some of the columns of an {\tt array} or {\tt tabular}
environment.  The command \hbox{\verb"\cline{"{\tt $i$-$j$}\verb|}|}
draws a line across columns $i$ through $j$ inclusive.  It is used
just like \hbox{\verb"\hline"}, except multiple \hbox{\verb"\cline"}
commands can be used to draw lines across distinct columns.
For example
\begin{verbatim}
  ... \\ \cline{1-3} \cline{5-5}
\end{verbatim}
Draws horizontal lines across columns 1-3 and column 5.


\subsubsection* {pages 117-8}
The \hbox{\verb"\newlist"} command has been eliminated.  In its
place are two commands \hbox{\verb"\newenvironment"} and
\hbox{\verb"\renewenvironment"} for defining arbitrary environments.
To understand how to use these commands, you must first realize that
\begin{verbatim}
       \begin{foo}  ...  \end{foo}
\end{verbatim}
is essentially translated to
\begin{verbatim}
       {\foo  ...  \endfoo}
\end{verbatim}
The command
\begin{verbatim}
      \newenvironment{foo}[3]{def1}{def2}
\end{verbatim}
acts the way you'd expect
\begin{verbatim}
     \newcommand{\foo}[3]{def1}\newcommand{\endfoo}{def2}
\end{verbatim}
to, except that \hbox{\verb"\newcommand"} won't allow you to define a
command name that begins with \hbox{\verb"\end"}.  (Of course, the
``{\tt [3]}'' represents an optional argument.)  The effect of
\hbox{\verb"\renewenvironment"} is similar.

\noindent
{\it The following paragraph is superceded by a change made in
version~2.08.}\\*
You will often want the \hbox{\verb"\begin{foo}"} to
execute the \hbox{\verb"\begin"} of some other envirnment and the
\hbox{\verb"\end{foo}"} to execute its \hbox{\verb"\end"}.  This
doesn't work.  The other environment must be called without using the
\hbox{\verb"\begin{...}"} and \hbox{\verb"\end{...}"} commands, as
shown below:
\begin{verbatim}
  WRONG:  \newenvironment{foo}{\begin{list}...}{\end{list}}
  RIGHT:  \newenvironment{foo}{\list...}{\endlist}
\end{verbatim}

\subsubsection*{pages 119 ff} 
A new document substyle {\tt leqno} now exists.  It causes equation
numbers to appear on the left instead of the right.  It works with
all styles and substyles, including {\tt fleqn}.

\subsubsection* {page 119}
There is now a document style for producing camera-ready output for
the ACM proceedings format---the large, two-column pages.  The {\tt
acm} substyle of the \hbox{\verb"article"} style produces output
formatted for large ($10\times14$) paper; the size known to the Imagen
printer as B4.  You should use the {\tt 12pt} substyle, so your
file should begin with
\begin{verbatim}
       \documentstyle[12pt,acm]{article}
\end{verbatim}
Use the \hbox{\verb"\maketitle"} command and the \hbox{\verb"abstract"}
environment to get the title and abstract in the right places.
You can leave a space at the bottom of the first column for the
copyright notice with the \hbox{\verb"\copyrightspace"} command.  It
normally follows right after the \hbox{\verb"\maketitle"}.  It works
by producing a blank footnote, so it must follow any commands that
produce footnotes in the first column.  The \hbox{\verb"\head"}
command will cause its argument to be printed as an identification at
the bottom of the page.

\subsubsection*{page 120}
A new command \hbox{\verb"\flushbottom"} has been added. It is
the inverse of the\linebreak %%
\hbox{\verb"\raggedbottom"} command introduced
in version~2.06.  It has been added because \hbox{\verb"\raggedbottom"}
is now the default in the \hbox{\verb"article"} and \hbox{\verb"report"}
styles unless the \hbox{\verb"twoside"} or \hbox{\verb"twocolumn"}
option is specified.


\subsubsection*{page 126} 
A new parameter \hbox{\verb"\arraystretch"} has been added to control
the interline spacing in the {\tt tabular} and {\tt array}
environments.  The spacing between rows will be \hbox{\verb"\arraystretch"} 
times the current value of \hbox{\verb"\baselineskip"}.

Note: The inter-row spacing in these environments is produced by a
putting a strut on each line rather than by \TeX's normal baseline
spacing mechanism.  The \hbox{\verb"\\[...]"} command (with a positive
argument) works by increasing the depth of this strut.  This can fail
to add the expected amount of extra space if there is something on the
line that extends further down than the strut.


\subsubsection*{page 134}
The argument of the \hbox{\verb"\index"} command may now contain
any character except a curly brace (\hbox{\verb"{"} or \hbox{\verb"}"}).
That argument will be written onto the {\tt .idx} file verbatim.
The same applies to the \hbox{\verb|\glossary|} command.

\subsubsection*{page 143 ff}
You may get an obscure error after \LaTeX\ has finished processing
your input file, when the message file indicates that it is reading
the {\tt .AUX} file.  Such an error is probably caused by a command in
the argument of a sectioning or \hbox{\verb"\caption"} command that
needs to be \hbox{\verb"\protect"}'ed.

\subsubsection*{page 146}
The error message 
\begin{verbatim}
       ! Can be used only in preamble.
\end{verbatim}
has been added.  It is caused by putting in the body of the text a
command that belongs in the preamble (before the
\hbox{\verb"\begin{document}"} command).

\subsubsection*{page 146}
The word \hbox{\tt preamble} has been replaced by 
``\hbox{\tt array arg}'' in the following three error messages:
\begin{verbatim}
       ! Illegal character in preamble.
       ! Missing p-arg in preamble.
       ! Missing @-exp in preamble.
\end{verbatim}
This was done only in later releases of Version~2.07. 

\section*{Changes in Version 2.08}


\subsubsection*{page 11, 71ff}

The \verb|\part| sectioning command has been added.  It is the highest
level of sectional unit and is optional.  A \verb|\part| command
does not change the numbering of chapters or sections, so the first
chapter of Part~II could be Chapter~12.  There is a \verb|\part*|
command that works like the {\tt *}-form of the other sectioning
commands.

\subsubsection*{page 28ff}

A font representing an unusual type-size/style combination may not
work right when used in math mode, either printing the wrong size
characters or not printing any characters and generating one of the
following error messages:%
\begin{verbatim}
       !  \textfont         ... is undefined (character ...).
       !  \scriptfont       ... is undefined (character ...).
       !  \scriptscriptfont ... is undefined (character ...).
\end{verbatim}
To solve this problem, 
use a command of the form
\begin{quote}
\verb|\load{|{\it size\/}\verb|}{|{\it style\/}\verb|}|
\end{quote}
where {\it size\/} is the size-changing command and {\it style\/} is
the type-style command that together specify the desired font.
The \hbox{\verb|\load|} command should come before the first use of
the font in math mode, and should not be inside braces or an environment.


\subsubsection*{page 32ff}

The command \hbox{\verb|\newfont{\foo}{|\it fontnm\verb|}|} defines
\hbox{\verb|\foo|} to be a declaration that chooses the font named
{\it fontnm}.  You can't use \hbox{\verb|\foo|} in math mode.

The command \hbox{\verb|\symbol{|\it char\verb|}|} produces the
symbol with character code {\it char\/} in the currently selected
font.  Octal character codes are prefaced by \hbox{\verb|'|} and
hexadecimal codes by \hbox{\verb|"|}, so you can type either
\hbox{\verb|\symbol{26}|}, or \hbox{\verb|\symbol{'32}|}, or
\hbox{\verb|\symbol{"1A}|}, since $26 = 32_{8} = 1A_{16}$.

\subsubsection*{pages 69,148}

The \hbox{\verb|figure*|} and \hbox{\verb|table*|} environments are
allowed in single-column format, being equivalent there to ordinary
\hbox{\verb|figure|} and \hbox{\verb|table|} environments, so they
produce no error message.  When writing a paper, even if you're using
a single-column format, it's a good idea to decide whether a figure or
table should be single- or double-column if it were in two-column
format.  That way, the right thing happens if you ever decide to
switch to double columms.

\subsubsection* {page 78ff}

The length \hbox{\verb|\fill|} has been added.  It is the ``rubber''
length added by an \hbox{\verb|\hfill|} command, so
\hbox{\verb|\hspace{\fill}|} is equivalent to \hbox{\verb|\hfill|}.
The command \hbox{\verb|\stretch{1.3}|} produces a rubber space of
length 1.3 times that produced by \hbox{\verb|\fill|}.  It can be used
only as the argument of \hbox{\verb|\hspace|} or \hbox{\verb|\vspace|}
(or their {\tt *}-forms), which now can produce rubber spaces.

\subsubsection*{page 81ff}

The command \hbox{\verb|\value{foo}|} produces the value of the
counter \hbox{\tt foo}.
It can be used where \LaTeX\ expects an integer or number,
such as the second argument of
a \hbox{\verb|\setcounter|} or \hbox{\verb|\addtocounter|}
command, or in
\begin{verbatim}
       \hspace{\value{foo}\parindent}
\end{verbatim}
It is useful for doing arithmetic with counters.

\subsubsection*{page 103ff}
Putting the command \hbox{\verb|\extracolsep{|{\it width\/}\verb|}|}
in an {\tt @}-expression causes an extra \hbox{\it width\/} of
space between all subsequent columns.  This is in addition to the
space normally put between the columns.  Unlike the normal intercolumn
space, this extra space between columns is not suppressed by
an {\tt @}-expression.

There is a new {\tt *}-form of the {\tt tabular} environment, which
has an extra initial argument that specifies the width of the
box that it produces.  When using the \hbox{\verb|tabular*|} environment,
you should use the \hbox{\verb|\extracolsep|} command in the argument
to add some rubber space (such as that provided by \hbox{\verb|\fill|})
to guarantee that the environment fills up the specified width.
For example,
\begin{verbatim}
  \begin{tabular*}{4in}[t]{ll@{\extracolsep{\fill}}lr@{\extracolsep
       {0in}\hspace{\tabcolsep}}r} ...
\end{verbatim}
produces a five-column environment that is four inches wide.
Extra space is added after the second and third columns to
fill out the four inch width, while there is a standard intercolumn
(\hbox{\verb"\tabcolsep"}) space after the first and fourth columns.

\subsubsection*{page 104}

Two \hbox{\verb|\hline|} commands in a row produce two lines with
a space between them.  The rules produced by \hbox{\tt |} characters
in the \hbox{\tt array} or \hbox{\tt tabular} environment's argument
are interrupted by this space.

\subsubsection* {pages 117-8}

{\it The following change is made obsolete in version 2.09.}

Two new commands \hbox{\verb"\envbegin"} and \hbox{\verb"\envend"}
have been added for use with the environment-defining commands
introduced in version~2.07.  A \hbox{\verb"\begin"} or \hbox{\verb"\end"}
command cannot invoke an unmatched \hbox{\verb"\begin"} or \hbox{\verb"\end"}
command.  Instead, it must invoke \hbox{\verb"\envbegin"} or
\hbox{\verb"\envend"}, as indicated in the following example.
\begin{verbatim}
  WRONG:  \newenvironment{foo}{\begin{list}...}{\end{list}}
  RIGHT:  \newenvironment{foo}{\envbegin{list}...}{\envend{list}}
\end{verbatim}

\subsubsection* {page 121}

The \hbox{\verb|\head|} command has been eliminated.
In its place are the commands \hbox{\verb|\markboth|} 
and \hbox{\verb|\markright|}
having the format
\begin{quote}
\verb|\markboth{|{\it left head\/}\verb|}{|{\it right head\/}\verb|}|\\
\verb|\markright{|{\it right head\/}\verb|}|
\end{quote}
for setting either both or just the right heading.  In addition to
their use with the \hbox{\tt myheadings} page style, you can use them
to override the normal headings in the \hbox{\tt headings} style,
since \LaTeX\ uses these same commands to generate those heads.  You
should note that a left-hand heading is generated by the last
\hbox{\verb|\markboth|} command before the end of the page, while a
right-hand heading is generated by the first \hbox{\verb|\markboth|}
or \hbox{\verb|\markright|} that comes on the page if there is one,
otherwise by the last one before the page.

\subsubsection* {page 137}

The \hbox{\verb|\cite|} command now has an optional argument that
adds a note to the citation.  For example, 
\begin{verbatim}
       \cite[pages 112-134]{jones:foo}
\end{verbatim}
might produce ``[34, pages 112-134]''.  This obviously shouldn't be
used with a \hbox{\verb|\cite|} command that produces multiple
citations like ``[12,34,45]''.

\section*{Changes in Version 2.09}

\subsubsection*{no relevant page}
The \hbox{\verb|\envbegin|} and \hbox{\verb|\envend|} commands
introduced in Version~2.07 have been eliminated. The arguments of
\hbox{\verb|\newenvironment|} can contain \hbox{\verb|\begin|} and
\hbox{\verb|\end|} commands.

\subsubsection*{no relevant page}

A new \hbox{\verb"\lefteqn"} command has been added, mainly for use in
an \hbox{\tt eqnarray} or \hbox{\tt eqnarray*} environment (introduced
in version~2.05), to help format wide equations that need to be broken
across lines.  It typesets its argument in display math style, but
treats it as if it had zero width when aligning columns.  For example,
you can produce
\begin{eqnarray*}
\lefteqn{\sum_{1}^{N} u_{i}+v_{i}+w_{i}=}\\
 & & a + b + c + d + e + f + g + h + i\\
 & & \mbox{}+ j + k + l + m + n + o + p
\end{eqnarray*}
by typing
\begin{verbatim}
\begin{eqnarray*}
\lefteqn{\sum_{1}^{N} u_{i}+v_{i}+w_{i}=}\\
 & & a + b + c + d + e + f + g + h + i\\
 & & \mbox{}+ j + k + l + m + n + o + p
\end{eqnarray*}
\end{verbatim}
(The \hbox{\verb"\mbox{}"} tells \TeX\ that the following $+$ is a binary
rather than a unary operator.)


\subsubsection* {page 28, top}

Eliminate the six lines beginning with \hbox{\verb"\hbox(6.94444"}.
All such printout has been eliminated from overfull and underfull box
messages.  (It will, however, appear in the log file.) This and all
other box descriptions can be made to appear on the terminal with the
\hbox{\verb"\showoverfull"} declaration.  (It obeys the normal scoping
rules.)

\subsubsection*{page 39}
Theorem-like environments now have an optional argument.  
If a \hbox{\verb"lemma"} environment has been defined with
a \hbox{\verb"\newtheorem"} command, then typing
\begin{verbatim}
      \begin{lemma}[Zorn]  For every ...
\end{verbatim}
will produce something like
\begin{quote}
  {\bf Lemma 4 (Zorn)} \ For every \ldots
\end{quote}

\subsubsection* {page 95, middle}

The \hbox{\verb"\sloppy"} command has been modified so it never
produces an overfull box.  (Well, hardly ever.)  There is also
a \hbox{\tt sloppypar} environment that typesets one or more
paragraphs with \hbox{\verb"\sloppy"} in effect.  (The environment
starts and ends a paragraph, so you don't have to worry about
leaving a blank line at the end as you do when using the
\hbox{\verb"\sloppy"} declaration.)


\subsubsection*{page 97}

There are many places, such as before and after displayed equations or
list environments, where the \hbox{\verb|\nopagebreak|} command has no
effect.  To overcome this problem, a new \hbox{\verb|\samepage|}
declaration has been added.  It inhibits page breaking almost anywhere
within its scope except between paragraphs, where
\hbox{\verb|\nopagebreak|} works nicely, or where a
\hbox{\verb|\pagebreak|} or \hbox{\verb|\nopagebreak|} command
explicitly allows breaking.  The rules determining where it applies
are a bit complicated; to a first approximation, it applies to an
entire paragraph if its scope includes the blank line ending
the paragraph.

\subsubsection*{page 137}

A \hbox{\verb|\nocite|} command has been added.  Like
\hbox{\verb|\cite|}, it tells Bib\TeX\ to add one or more
references to the reference list, but it puts nothing in the
text.  It must come after the \hbox{\verb|\begin{document}|}
command.

\end{document}
