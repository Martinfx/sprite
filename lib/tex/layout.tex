% -*-LaTeX-*-
% <BEEBE.F77ANA>PAGE.LTX.22, 16-Oct-86 14:13:02, Edit by BEEBE
% <BEEBE.F77ANA>PAGE.LTX.1, 13-Oct-86 17:04:39, Edit by BEEBE
% Make pictures showing LaTeX page layouts
%-----------------------------------------------------------------------
% EVERYTHING TO THE RIGHT OF A  %  IS A REMARK TO YOU AND IS IGNORED BY
% LaTeX.
%
% WARNING!  DO NOT TYPE ANY OF THE FOLLOWING 10 CHARACTERS AS NORMAL TEXT
% CHARACTERS:
%                &   $   #   %   _    {   }   ^   ~   \
%
% The following seven are printed by typing a backslash in front of them:
%  \$  \&  \#  \%  \_  \{  and  \}.
%-----------------------------------------------------------------------
\newcommand{\X}[1]{{#1}\index{{#1}}}
\documentstyle[draft,ifthen]{article}

% ----------------------------------------------------------------------
%
% TIME OF DAY
%
\newcount\hh
\newcount\mm
\mm=\time
\hh=\time
\divide\hh by 60
\divide\mm by 60
\multiply\mm by 60
\mm=-\mm
\advance\mm by \time
\def\hhmm{\number\hh:\ifnum\mm<10{}0\fi\number\mm}

% NB: For computed dimension parameters, we cannot use
% \newcommand{}, because this expands to a TeX \def which does
% not evaluate the definition before assigning it to the control
% sequence name; we use \xdef directly to force evaluation
\newcount\T      % temporary counter for arithmetic calculations
\T=0

% **********************************************************************
% WARNING: Do not insert ANY aditional whitespace in these
% macros--otherwise it ends up in the TeX boxes and ruins the
% positioning, sigh....
% **********************************************************************

% NAMEBOX{x}{y}{dx}{dy}{width}{height}{pos}{label} -- framed label with
% box lower-left corner at (x+dx,y+dy)
\newcommand{\NAMEBOX}[8]{\put(#1,#2){\begin{picture}(0,0)(-#3,-#4)\ignorespaces
    \framebox(#5,#6)[#7]{#8}\end{picture}}}

% HARROW{x}{y}{dx}{dy}{length}{side}{pos}{label} -- horizontal
% labeled arrow with left point at (x+dx,y+dy), label on bottom
% (side=b) or top (side=t) of arrow, in makebox[pos]
\newcommand{\HARROW}[8]{\put(#1,#2){\begin{picture}(0,0)(-#3,-#4)\ignorespaces
    \put(0,0){\vector(1,0){#5}}\ignorespaces
    \put(#5,0){\vector(-1,0){#5}}\ignorespaces
    \ifthenelse{\equal{#6}{b}}{\ignorespaces
      \put(0,-\TAD){\makebox(#5,0)[#7]{#8}}
    }{\ignorespaces
    \ifthenelse{\equal{#6}{t}}{}{\ignorespaces
      \typeout{Side #6 must be ``b'' or ``t''--``t'' assumed}}\ignorespaces
      \put(0,\TAD){\makebox(#5,0)[#7]{#8}}
    }
    \end{picture}}}

% VARROW{x}{y}{dx}{dy}{length}{side}{pos}{label} -- vertical
% labeled arrow with left point at (x+dx,y+dy), label on left
% (side=l) or right (side=r) of arrow, in makebox[pos]
\newcommand{\VARROW}[8]{\put(#1,#2){\begin{picture}(0,0)(-#3,-#4)\ignorespaces
    \put(0,0){\vector(0,1){#5}}\ignorespaces
    \put(0,#5){\vector(0,-1){#5}}\ignorespaces
    \ifthenelse{\equal{#6}{l}}{\ignorespaces
      \put(-\TAD,0){\makebox(0,#5)[#7]{#8}}
    }{\ignorespaces
      \ifthenelse{\equal{#6}{r}}{}{\ignorespaces
        \typeout{Side #6 must be ``r'' or ``l''--``r'' assumed}}\ignorespaces
        \put(\TAD,0){\makebox(0,#5)[#7]{#8}}
    }
    \end{picture}}}

% VRULE{x}{y}{dx}{dy}{length}{side}{pos}{label} -- vertical
% rule with left point at (x+dx,y+dy), label on left
% (side=l) or right (side=r) of rule, in makebox[pos]
\newcommand{\VRULE}[8]{\put(#1,#2){\begin{picture}(0,0)(-#3,-#4)\ignorespaces
    \put(0,0){\line(0,1){#5}}\ignorespaces
    \ifthenelse{\equal{#6}{l}}{\ignorespaces
      \put(0,0){\makebox(0,#5)[#7]{$\rightarrow${}#8}}
    }{\ignorespaces
      \ifthenelse{\equal{#6}{r}}{}{\ignorespaces
        \typeout{Side #6 must be ``r'' or ``l''--``r'' assumed}}\ignorespaces
        \put(0,0){\makebox(0,#5)[#7]{$\leftarrow${}#8}}
    }
  \end{picture}}}
% **********************************************************************
% If you wish to change any of these values:
%       * definitions are mostly in alphabetical order
%       * remember units are in centipoints for accurate
%         positioning
%       * most parameters are taken directly from values in
%         BK11.STY, but in order to improve the appearance of the
%         figure, a few have been increased to avoid overlap of
%         labels and/or boxes, or to loosen up the figure
%       * ALL parameters which depend on others are defined by
%         computation and \xdef's below; they must NEVER be
%         changed
%
% For 11pt BOOK style, \textwidth / \pagewidth(8.5in) = 0.5294
% Anything larger than this will generate an overfull box.  We
% choose 0.50 since that makes scaling trivial for the reader
\newcommand{\UNITLENGTH}{0.005pt}
\newcommand{\SCALEFACTOR}{50\%}
%
% US papersize...
%
\newcommand{\PAGEHEIGHT}{79497}         % 11in in centipoints
\newcommand{\PAGEWIDTH}{61430}          % 8.5in in centipoints
%
% Typical dimensions from BK10.DOC
%
\newcommand{\BASELINESKIP}{2000}        % really 1000, but this is
                                        % too tight for the figure
\newcommand{\COLUMNSEP}{1000}
\newcommand{\COLUMNSEPRULE}{0}

% DVI drivers put top left corner at (1in,1in) from physical page
% left corner
\newcommand{\DVIXOFFSET}{7227}
\newcommand{\DVIYOFFSET}{7227}

\newcommand{\EVENSIDEMARGIN}{10841}

\newcommand{\FOOTHEIGHT}{2400} % really 1200, but too tight
\newcommand{\FOOTNOTESEP}{665}
\newcommand{\FOOTSKIP}{4207}    % really 2529, but too tight; for symmetry,
                                % make FOOTSKIP=HEADSEP+FOOTHEIGHT
\newcommand{\HEADHEIGHT}{2400}  % really 1200, but too tight
\newcommand{\HEADSEP}{1807}

\newcommand{\MARGINPARPUSH}{2000}       % really 500, but too tight
\newcommand{\MARGINPARSEP}{2800}        % really 700, but too tight
\newcommand{\MARGINPARWIDTH}{7227}
\newcommand{\MARGINNOTEHEIGHT}{4000}    % arbitrary value (holds
                                        % ``Margin Note A'' on
                                        % 2 lines)

\newcommand{\ODDSIDEMARGIN}{3613}
%
% Upper left page corner (0,0) is at (1in,1in) on physical page
% for DVI drivers.  Position the LOWER left corner at
% (\PAGEXORIGIN,\PAGEYORIGIN), where \PAGEYORIGIN is computed
% below
\newcommand{\PAGEXORIGIN}{\DVIXOFFSET}

\newcommand{\TAD}{800}     % how far to move labels from their arrows
\newcommand{\TEXTHEIGHT}{50400}
\newcommand{\TEXTWIDTH}{32522}
\newcommand{\TOPMARGIN}{5420}

% **********************************************************************
% DO NOT CHANGE any of these computed parameters
%

%
\T=\TEXTWIDTH
\advance\T by -\COLUMNSEP
\divide\T by 2
\xdef\COLUMNWIDTH{\the\T}

\T=\TEXTWIDTH
\divide\T by 2
\advance\T by \ODDSIDEMARGIN
\xdef\COLUMNXMIDDLE{\the\T}     % 0.5*TEXTWIDTH+ODDSIDEMARGIN

\T=\COLUMNWIDTH
\advance\T by \ODDSIDEMARGIN
\advance\T by \COLUMNSEP
\xdef\COLUMNTWOX{\the\T}        % COLUMNWIDTH+COLUMNSEP+ODDSIDEMARGIN

\T=\FOOTSKIP
\advance\T by \TEXTHEIGHT
\advance\T by \HEADSEP
\xdef\HEADYORIGIN{\the\T}           % FOOTSKIP+TEXTHEIGHT+HEADSEP
%
% Inner TeX page dimensions are (\INNERWIDTH,\INNERHEIGHT).  All TeX
% coordinates are relative to (0,0) at upper-left corner of this
% page, although for the figures, we put (0,0) at lower-left
% corner.
%
\T=\TOPMARGIN
\advance\T by \HEADHEIGHT
\advance\T by \HEADSEP
\advance\T by \TEXTHEIGHT
\advance\T by \FOOTSKIP
% INNERHEIGHT=TOPMARGIN+HEADHEIGHT+HEADSEP+TEXTHEIGHT+FOOTSKIP
\xdef\INNERHEIGHT{\the\T}

\T=\ODDSIDEMARGIN
\advance\T by \TEXTWIDTH
\advance\T by \EVENSIDEMARGIN
\xdef\INNERWIDTH{\the\T}  % ODDSIDEMARGIN+TEXTWIDTH+EVENSIDEMARGIN

\T=\ODDSIDEMARGIN
\advance\T by \TEXTWIDTH
\advance\T by \MARGINPARSEP
\xdef\MARGINNOTEXORIGIN{\the\T} % ODDSIDEMARGIN+TEXTWIDTH+MARGINPARSEP

\T=\TEXTHEIGHT
\multiply\T by 75
\divide\T by 100
\advance\T by \FOOTSKIP
\xdef\MARGINNOTEYA{\the\T}      % .75*TEXTHEIGHT+FOOTSKIP

\T=\MARGINNOTEYA
\advance\T by -\MARGINNOTEHEIGHT
\advance\T by -\MARGINPARPUSH
\xdef\MARGINNOTEYB{\the\T}      % MARGINNOTEYA-MARGINNOTEHEIGHT-MARGINPARPUSH

\T=\MARGINNOTEYA
\advance\T by \MARGINNOTEHEIGHT
\xdef\MARGINNOTEYC{\the\T}      % MARGINNOTEYA+MARGINNOTEHEIGHT

\T=\PAGEHEIGHT
\advance\T by -\DVIYOFFSET
\xdef\PAGEYTOP{\the\T}          % PAGEHEIGHT-DVIYOFFSET

\T=\PAGEYTOP
\advance\T by -\INNERHEIGHT
\xdef\PAGEYORIGIN{\the\T}       % PAGEYTOP-INNERHEIGHT

\T=\ODDSIDEMARGIN
\advance\T by \TEXTWIDTH
\xdef\RIGHTEDGE{\the\T}         % ODDSIDEMARGIN+TEXTWIDTH

\T=\TEXTHEIGHT
\multiply\T by 4
\divide\T by 10
\xdef\TEXTY{\the\T}             % TEXTY = 0.4*TEXTHEIGHT (we put
                                % sample text here)

\T=\TEXTHEIGHT
\divide\T by 3
\xdef\TEXTWIDTHY{\the\T}        % TEXTWIDTHY = TEXTHEIGHT/3 (we
                                % put \textwidth label here)
\begin{document}
  \begin{figure}
 \caption[\string\LaTeX{} single-column page layout]{\LaTeX{}
 single-column page layout.  The actual proportions correspond to
 parameter values in the 11pt BOOK document style.
 Note that standard-conforming DVI drivers are required to place
 the \TeX{} upper-left page corner one inch over and down from
 the corner of the physical output page.
 This figure is scaled to \SCALEFACTOR{} of actual page size.
 It was produced on \today{} at \hhmm.}
    \begin{center}
      \begin{small}       % make text somewhat smaller
      \setlength{\unitlength}{\UNITLENGTH}
        \begin{picture}(\PAGEWIDTH,\PAGEHEIGHT)   % sizes in centipoints

%
          \NAMEBOX{0}{0}{0}{0}{\PAGEWIDTH}{\PAGEHEIGHT}{}{}
%
          \HARROW{0}{\PAGEYTOP}{0}{0}{\PAGEXORIGIN}{b}{}{1in}
%
          \VARROW{\PAGEXORIGIN}{\PAGEYTOP}{0}{0}{\DVIYOFFSET}{r}{l}{1in}
%
          \put(\PAGEXORIGIN,\PAGEYORIGIN){
          \begin{picture}(\INNERWIDTH,\INNERHEIGHT)
            % Draw 3-sided inner page frame--no bottom side
            % because it is not significant for dimensioning
            \put(0,0){\line(0,1){\INNERHEIGHT}}
            \put(0,\INNERHEIGHT){\line(1,0){\INNERWIDTH}}
            \put(\INNERWIDTH,0){\line(0,1){\INNERHEIGHT}}
%
            \NAMEBOX{\ODDSIDEMARGIN}{\TEXTY}{0}{0}{\TEXTWIDTH}{
                \BASELINESKIP}{l}{A line of text\ldots}
            \NAMEBOX{\ODDSIDEMARGIN}{\TEXTY}{0}{-\BASELINESKIP}{
                \TEXTWIDTH}{\BASELINESKIP}{l}{Next line of text\ldots}
%
            \VARROW{\COLUMNXMIDDLE}{\TEXTY}{0}{-\BASELINESKIP}{
              \BASELINESKIP}{r}{l}{\tt\string\baselinestretch
              $\times$\string\baselineskip}
%
            \HARROW{0}{\TEXTHEIGHT}{0}{0}{\ODDSIDEMARGIN}{t}{}{
              \tt\string\oddsidemargin}
            \HARROW{0}{\TEXTHEIGHT}{0}{0}{\ODDSIDEMARGIN}{b}{}{
              \tt\string\evensidemargin}
            %
            % Page text label
            \NAMEBOX{\ODDSIDEMARGIN}{\FOOTSKIP}{0}{0}{\TEXTWIDTH}{
              \TEXTHEIGHT}{}{Page Text}
            % Page footer box, arrow, and label
            \NAMEBOX{\ODDSIDEMARGIN}{0}{0}{0}{\TEXTWIDTH}{
              \FOOTHEIGHT}{}{Page Footer}
            %
            \VARROW{\ODDSIDEMARGIN}{0}{-\TAD}{0}{\FOOTHEIGHT}{l}{r}{
              \tt\string\footheight}
            %
            % Footskip arrow and label
            \VARROW{\RIGHTEDGE}{0}{\TAD}{0}{\FOOTSKIP}{r}{l}{
              \tt\string\footskip}
            %
            \HARROW{\ODDSIDEMARGIN}{\TEXTWIDTHY}{0}{0}{\TEXTWIDTH}{b}{}{
              \tt\string\textwidth}
            %
            \VARROW{\ODDSIDEMARGIN}{\FOOTSKIP}{-\TAD}{0}{
              \TEXTHEIGHT}{l}{r}{\tt\string\textheight$\rightarrow$}
            %
            \VARROW{\ODDSIDEMARGIN}{\HEADYORIGIN}{-\TAD}{0}{
              \HEADHEIGHT}{l}{r}{\tt\string\headheight}
            %
            \VARROW{\RIGHTEDGE}{\HEADYORIGIN}{\TAD}{-\HEADSEP}{
              \HEADSEP}{r}{l}{\tt\string\headsep}
            %
            \VARROW{\RIGHTEDGE}{\INNERHEIGHT}{\TAD}{-\TOPMARGIN}{
              \TOPMARGIN}{r}{l}{\tt\string\topmargin}
            %
            \NAMEBOX{\ODDSIDEMARGIN}{\HEADYORIGIN}{0}{0}{\TEXTWIDTH}{
              \HEADHEIGHT}{}{Page Header}
            %
            \NAMEBOX{\MARGINNOTEXORIGIN}{\MARGINNOTEYA}{0}{0}{
              \MARGINPARWIDTH}{\MARGINNOTEHEIGHT}{}{\shortstack{
              Margin\\note A}}
            %
            \VARROW{\MARGINNOTEXORIGIN}{\MARGINNOTEYA}{-\TAD}{
              -\MARGINPARPUSH}{\MARGINPARPUSH}{l}{r}{
            \tt\string\marginparpush}
            %
            \NAMEBOX{\MARGINNOTEXORIGIN}{\MARGINNOTEYB}{0}{0}{
              \MARGINPARWIDTH}{\MARGINNOTEHEIGHT}{}{\shortstack{Margin\\note B}
}
            %
            \HARROW{\MARGINNOTEXORIGIN}{\MARGINNOTEYB}{0}{-\TAD}{
              \MARGINPARWIDTH}{b}{}{\tt\string\marginparwidth}
            %
            \HARROW{\MARGINNOTEXORIGIN}{\MARGINNOTEYC}{
              -\MARGINPARSEP}{\TAD}{\MARGINPARSEP}{t}{}{
              \tt\string\marginparsep}
            %
          \end{picture}}
        \end{picture}
      \end{small}
    \end{center}
  \end{figure}
            \clearpage
  \begin{figure}
 \caption[\string\LaTeX{} double-column page layout]{\LaTeX{}
 double-column page layout.  The actual proportions correspond to
 parameter values in the 11pt BOOK document style.
 Note that standard-conforming DVI drivers are required to place
 the \TeX{} upper-left page corner one inch over and down from
 the corner of the physical output page.
 This figure is scaled to \SCALEFACTOR{} of actual page size.
 It was produced on \today{} at \hhmm.}
    \begin{center}
      \begin{small}       % make text somewhat smaller
      \setlength{\unitlength}{\UNITLENGTH}
        \begin{picture}(\PAGEWIDTH,\PAGEHEIGHT)   % sizes in centipoints
            \NAMEBOX{0}{0}{0}{0}{\PAGEWIDTH}{\PAGEHEIGHT}{}{}
            %
            \HARROW{0}{\PAGEYTOP}{0}{0}{\PAGEXORIGIN}{b}{}{1in}
            %
            \VARROW{\PAGEXORIGIN}{\PAGEYTOP}{0}{0}{\DVIYOFFSET}{r}{l}{1in}
            %
            \put(\PAGEXORIGIN,\PAGEYORIGIN){
          \begin{picture}(\INNERWIDTH,\INNERHEIGHT)
            %
            % Draw 3-sided inner page frame--no bottom side
            % because it is not significant for dimensioning
            \put(0,0){\line(0,1){\INNERHEIGHT}}
            \put(0,\INNERHEIGHT){\line(1,0){\INNERWIDTH}}
            \put(\INNERWIDTH,0){\line(0,1){\INNERHEIGHT}}
            %
            \NAMEBOX{\ODDSIDEMARGIN}{\TEXTY}{0}{0}{\COLUMNWIDTH}{
              \BASELINESKIP}{l}{A line of text\ldots}
            \NAMEBOX{\ODDSIDEMARGIN}{\TEXTY}{0}{-\BASELINESKIP}{
              \COLUMNWIDTH}{\BASELINESKIP}{l}{Next line of text\ldots}
            %
            \HARROW{0}{\TEXTHEIGHT}{0}{0}{\ODDSIDEMARGIN}{t}{}{
              \tt\string\oddsidemargin}
            \HARROW{0}{\TEXTHEIGHT}{0}{0}{\ODDSIDEMARGIN}{b}{}{
              \tt\string\evensidemargin}
            %
            % Page text label
            %
            \NAMEBOX{\ODDSIDEMARGIN}{\FOOTSKIP}{0}{0}{\COLUMNWIDTH}{
              \TEXTHEIGHT}{}{}
            %
            \NAMEBOX{\COLUMNTWOX}{\FOOTSKIP}{0}{0}{\COLUMNWIDTH}{
              \TEXTHEIGHT}{}{}
            %
            \VRULE{\COLUMNXMIDDLE}{\FOOTSKIP}{0}{0}{\TEXTHEIGHT}{r}{l}{
              rule width is \tt\string\columnseprule}
            %
            \VARROW{\COLUMNXMIDDLE}{\TEXTY}{-\COLUMNSEP}{
              -\BASELINESKIP}{\BASELINESKIP}{r}{l}{
              \tt\string\baselinestretch $\times$\string\baselineskip}
            %
            \HARROW{\COLUMNTWOX}{\FOOTSKIP}{-\COLUMNSEP}{-\TAD}{\COLUMNSEP}{b}{
}{
              \tt\string\columnsep}
            %
            % Page footer box, arrow, and label
            \NAMEBOX{\ODDSIDEMARGIN}{0}{0}{0}{\TEXTWIDTH}{
              \FOOTHEIGHT}{}{Page Footer}
            %
            \VARROW{\ODDSIDEMARGIN}{0}{-\TAD}{0}{
              \FOOTHEIGHT}{l}{r}{\tt\string\footheight}
            %
            % Footskip arrow and label
            \VARROW{\RIGHTEDGE}{0}{\TAD}{0}{\FOOTSKIP}{r}{l}{
              \tt\string\footskip}
            %
            \HARROW{\ODDSIDEMARGIN}{\TEXTWIDTHY}{0}{0}{\TEXTWIDTH}{b}{}{
              \tt\string\textwidth}
            %
            \VARROW{\ODDSIDEMARGIN}{\FOOTSKIP}{-\TAD}{0}{
              \TEXTHEIGHT}{l}{r}{\tt\string\textheight$\rightarrow$}
            %
            \VARROW{\ODDSIDEMARGIN}{\HEADYORIGIN}{-\TAD}{0}{
              \HEADHEIGHT}{l}{r}{\tt\string\headheight}
            %
            \VARROW{\RIGHTEDGE}{\HEADYORIGIN}{\TAD}{-\HEADSEP}{
              \HEADSEP}{r}{l}{\tt\string\headsep}
            %
            \VARROW{\RIGHTEDGE}{\INNERHEIGHT}{\TAD}{-\TOPMARGIN}{
              \TOPMARGIN}{r}{l}{\tt\string\topmargin}
            %
            \NAMEBOX{\ODDSIDEMARGIN}{\HEADYORIGIN}{0}{0}{\TEXTWIDTH}{
              \HEADHEIGHT}{}{Page Header}
            %
            \NAMEBOX{\MARGINNOTEXORIGIN}{\MARGINNOTEYA}{0}{0}{
              \MARGINPARWIDTH}{\MARGINNOTEHEIGHT}{}{\shortstack{
              Margin\\note A}}
            %
            \VARROW{\MARGINNOTEXORIGIN}{\MARGINNOTEYA}{-\TAD}{
              -\MARGINPARPUSH}{\MARGINPARPUSH}{l}{r}{
              \tt\string\marginparpush}
            %
            \NAMEBOX{\MARGINNOTEXORIGIN}{\MARGINNOTEYB}{0}{0}{
              \MARGINPARWIDTH}{\MARGINNOTEHEIGHT}{}{\shortstack{
              Margin\\note B}}
            %
            \HARROW{\MARGINNOTEXORIGIN}{\MARGINNOTEYB}{0}{-\TAD}{
              \MARGINPARWIDTH}{b}{}{\tt\string\marginparwidth}
            %
            \HARROW{\MARGINNOTEXORIGIN}{\MARGINNOTEYC}{
              -\MARGINPARSEP}{\TAD}{\MARGINPARSEP}{t}{}{
              \tt\string\marginparsep}
            %
          \end{picture}}
        \end{picture}
      \end{small}
    \end{center}
  \end{figure}
  \clearpage
 \begin{figure}
   \caption{Page layout parameters (in centipoints) for preceding figures.
   It was produced on \today{} at \hhmm.}
%
   \begin{center}
     \begin{tabular}{|l|r|}
        \hline
        {\tt\string\BASELINESKIP}       & \BASELINESKIP\\
        {\tt\string\COLUMNSEPRULE}      & \COLUMNSEPRULE\\
        {\tt\string\COLUMNSEP}          & \COLUMNSEP\\
        {\tt\string\COLUMNTWOX}         & \COLUMNTWOX\\
        {\tt\string\COLUMNXMIDDLE}      & \COLUMNXMIDDLE\\
        {\tt\string\DVIXOFFSET}         & \DVIXOFFSET\\
        {\tt\string\DVIYOFFSET}         & \DVIYOFFSET\\
        {\tt\string\EVENSIDEMARGIN}     & \EVENSIDEMARGIN\\
        {\tt\string\FOOTHEIGHT}         & \FOOTHEIGHT\\
        {\tt\string\FOOTNOTESEP}        & \FOOTNOTESEP\\
        {\tt\string\FOOTSKIP}           & \FOOTSKIP\\
        {\tt\string\HEADHEIGHT}         & \HEADHEIGHT\\
        {\tt\string\HEADSEP}            & \HEADSEP\\
        {\tt\string\HEADYORIGIN}        & \HEADYORIGIN\\
        {\tt\string\INNERHEIGHT}        & \INNERHEIGHT\\
        {\tt\string\INNERWIDTH}         & \INNERWIDTH\\
        {\tt\string\MARGINNOTEHEIGHT}   & \MARGINNOTEHEIGHT\\
        {\tt\string\MARGINNOTEXORIGIN}  & \MARGINNOTEXORIGIN\\
        {\tt\string\MARGINNOTEYA}       & \MARGINNOTEYA\\
        {\tt\string\MARGINNOTEYB}       & \MARGINNOTEYB\\
        {\tt\string\MARGINPARPUSH}      & \MARGINPARPUSH\\
        {\tt\string\MARGINPARSEP}       & \MARGINPARSEP\\
        {\tt\string\MARGINPARWIDTH}     & \MARGINPARWIDTH\\
        {\tt\string\ODDSIDEMARGIN}      & \ODDSIDEMARGIN\\
        {\tt\string\PAGEHEIGHT}         & \PAGEHEIGHT\\
        {\tt\string\PAGEWIDTH}          & \PAGEWIDTH\\
        {\tt\string\PAGEXORIGIN}        & \PAGEXORIGIN\\
        {\tt\string\PAGEYORIGIN}        & \PAGEYORIGIN\\
        {\tt\string\PAGEYTOP}           & \PAGEYTOP\\
        {\tt\string\RIGHTEDGE}          & \RIGHTEDGE\\
        {\tt\string\TAD}                & \TAD\\
        {\tt\string\TEXTHEIGHT}         & \TEXTHEIGHT\\
        {\tt\string\TEXTWIDTH}          & \TEXTWIDTH\\
        {\tt\string\TEXTWIDTHY}         & \TEXTWIDTHY\\
        {\tt\string\TOPMARGIN}          & \TOPMARGIN\\
        \hline
     \end{tabular}
   \end{center}
 \end{figure}
\clearpage
\end{document}
