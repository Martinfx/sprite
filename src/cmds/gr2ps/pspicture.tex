%
%  (c) 1986 John Coker
%  University of California, Berkeley
%
%  pspicture.tex - example TeX document with a PostScript illustration
%
\magnification=\magstep1
\baselineskip=14pt plus1pt
\parskip=5pt plus2pt
\nopagenumbers

\def\pspicture#1#2{\indent\vskip#2\special{#1}}
\def\pscaption#1{\hbox to \hsize{\hss{#1}\hss}}

\def\\{\char92 }
\def\{{\char123 }
\def\}{\char125 }

\font\sc=amr8
\def\PostScript{P{\sc OST}S{\sc CRIPT}}

This is a demonstration of a {\PostScript} picture inside \TeX.  We've
done all this neatly with a simple {\TeX} macro that imbeds a
{\tt \\special} command containing the name of a {\PostScript}
file that is to be used as the picture itself.  Of course, this
depends on the printer {\tt dvi} driver knowing that arguments to
{\tt \\special}s are names of {\PostScript} files.

And now, of course, it's time for an example.
The picture of the tank below was set into this text with the 
simple set of commands:

\settabs8\columns
\begingroup\baselineskip=12pt
\+&\tt\\midinsert\cr
\+&\tt\\pspicture\{tank.ps\}\{2in\}\cr
\+&\tt\\pscaption\{\\bf A PostScript Picture\}\cr
\+&\tt\\endinsert\cr
\endgroup

\noindent And here is the picture:

\midinsert
\pspicture{/d/texp/john/src/postscript/gr2ps/tank.ps}{2in}
\pscaption{\bf A PostScript Picture}
\endinsert

\noindent And here is text immediately following the picture.  Note
that these macros depend on the user correctly specifying the picture
height.  Unfortunately there is no way for {\TeX} and {\PostScript} to
interact on a level where {\TeX} could automatically know the height
of a {\PostScript} picture.

\medskip
\+&&&&John Coker\cr
\+&&&&August 1, 1986\cr

\bye
